The lexer is the lowest layer in the Nuua system as shown in \autoref{fig:nuua_system}. The lexer job is to transform
the program source code into a list of tokens. This step is also known as lexical analysis or tokenization.

\begin{figure}[H]
    \centering
    \begin{tikzpicture}
        [node distance=1cm]

        % Nodes of the layered system
        \node[block] (input) {List of characters};
        \node[block,right=of input] (lexer) {Lexer};
        \node[block,right=of lexer] (output) {List of tokens};

        % Lines
        \draw[line] (input) -- (lexer);
        \draw[line] (lexer) -- (output);

    \end{tikzpicture}

    % Caption and Label
    \caption{Lexer overview}
    \label{fig:lexer_overview}
\end{figure}

The lexer needs to provide an API to the layer above and one of the parameters of the scan function is the path of the
source file to scan. The lexer is also responsible for opening the file and creating a \texttt{char} array with its contents. This can be
done in the lexer class constructor seen in \autoref{ls:lexer_class}.
When the layer above wishes to scan the tokens the tokenization begins and therefore a list of tokens is built and returned.

\section{Strategy}

The strategy of the lexer is to use a state machine to determine the pattern of the tokens and be able to identify
what token type it's being scanned. The state machine finds a pattern and creates a specific token depending on
the state it landed. The state machine can also lead to no pattern. If there's a sequence of characters that don't match
a pattern, an error is then thrown using the logger module as described in \autoref{sec:logger}.

The implementation uses a C++ \texttt{switch} statement to match the first character of the pattern and further continues if needed
depending on the patterns.

In case a \texttt{\textbackslash n} is found, the lexer line is incremented and the column is reset to $0$. Since the current line and column
are properties in the lexer instance they only affect a single instance allowing multiple instances to scan different files at the same
time. \autoref{fig:finite_state_strings} shows an example pattern found in the lexer.

\begin{figure}[H]
	\centering
    \begin{subfigure}{\textwidth}
		\centering
        \begin{tikzpicture}[node distance=3cm]
            % Nodes
            \node[state] (1) {\texttt{S1}};
            \node[state, below left of=1] (2) {\texttt{S2}};
            \node[state, below right of=1] (3) {\texttt{S3}};
            % Lines
            \draw[line] (1) edge[left, bend right] node[xshift=-0.15cm, yshift=0.15cm] {\texttt{e1}} (2);
            \draw[line] (2) edge[loop left] node[xshift=-0.15cm] {\texttt{e2}} (2);
            \draw[line] (2) edge[below, bend right] node[yshift=-0.15cm] {\texttt{e3}} (3);
            \draw[line] (3) edge[right, bend right] node[xshift=0.15cm, yshift=0.15cm] {\texttt{e4}} (1);
        \end{tikzpicture}
		\caption{Diagram}
	\end{subfigure}
	\begin{subtable}{0.45\textwidth}
		\centering
        \begin{tabular}{ l p{4.5cm} }
            \textbf{State} & \textbf{Explanation} \\
            \texttt{S1} & Initial state of the state machine. \\
            \texttt{S2} & Build string. \texttt{s += c} \\
            \texttt{S3} & Create token \\
		\end{tabular}
		\caption{State table}
	\end{subtable}
	\begin{subtable}{0.45\textwidth}
		\centering
        \begin{tabular}{ l l l }
            \textbf{Event} & \textbf{Condition} & \textbf{Action} \\
            \texttt{e1} & \texttt{c == '"'} & \texttt{c = next\_char()} \\
            \texttt{e2} & \texttt{c != '"'} & \texttt{c = next\_char()} \\
            \texttt{e3} & \texttt{c == '"'} & \texttt{-} \\
            \texttt{e4} & \texttt{-} & \texttt{c = next\_char()} \\
		\end{tabular}
		\caption{Event table}
	\end{subtable}
	\caption{Example finite state machine for strings}
    \label{fig:finite_state_strings}
\end{figure}

As shown in \autoref{fig:scanner}, an efficient way to scan the file is using two pointers to some of the \texttt{chars} found in the source file.
The first pointer points to the start of the token being scanned and the second one advances accordingly with the help of the
state machine. When the state machine determines that a new token must be created the start pointer is used as the first character
pointer in the token as explained in \autoref{sec:tokens} and the length of the token can be determined by using pointer arithmetic given:

\begin{center}
$\texttt{length} = \texttt{current\_ptr} - \texttt{start\_ptr} + 1$
\end{center}

Since the string values can ignore the two \texttt{'"'} found at the
start and at the end of the token, the length will be decremented by two and the start pointer is incremented by one to get only the string value.
When a blank space is found in the initial state of the state machine the \texttt{start\_ptr} pointer address is then set to the \texttt{current\_ptr} address
and the current parser column increases.

\begin{figure}[H]
    \centering
    \begin{tikzpicture}[node distance=1cm]
        % Nodes
        \node[square, minimum height=0.75cm, minimum width=1cm,] (n1) {...};
        \node[square, minimum height=0.75cm, minimum width=1cm, right of=n1] (n2) {\texttt{"}};
        \node[square, minimum height=0.75cm, minimum width=1cm, right of=n2] (n3) {\texttt{a}};
        \node[square, minimum height=0.75cm, minimum width=1cm, right of=n3] (n4) {\texttt{b}};
        \node[square, minimum height=0.75cm, minimum width=1cm, right of=n4] (n5) {\texttt{c}};
        \node[square, minimum height=0.75cm, minimum width=1cm, right of=n5] (n6) {\texttt{"}};
        \node[square, minimum height=0.75cm, minimum width=1cm, right of=n6] (n7) {...};
        % Lines
        \draw[line] ($(n2.south) + (0, -0.75)$) node[anchor=north] {\texttt{start\_ptr}} -- (n2);
        \draw[line] ($(n6.south) + (0, -0.75)$) node[anchor=north] {\texttt{current\_ptr}} -- (n6);
    \end{tikzpicture}
    \caption{Lexer scan technique}
    \label{fig:scanner}
\end{figure}

\section{Tokens}
\label{sec:tokens}

Tokens represent specific lexeme patterns. The list of the patterns and the tokens associated with it can be seen on \autoref{fig:nuua_tokens_1},
\autoref{fig:nuua_tokens_2} and \autoref{fig:nuua_tokens_3}. The tokens must also take into consideration escaping characters. For example, a string
representation might be willing to use the char \texttt{'"'} as part of the string value and not determine the end of the string. Therefore, there
is a list of escaped chars that are taken into consideration as shown in \autoref{sec:espaced_chars}.

\begin{table}[H]
	\centering
	\begin{subtable}{0.45\textwidth}
		\centering
        \begin{tabular}{ l l }
            \textbf{Pattern} & \textbf{Token} \\
            \texttt{\textbackslash n} & \texttt{TOKEN\_NEW\_LINE} \\
            \texttt{(} & \texttt{TOKEN\_LEFT\_PAREN} \\
            \texttt{)} & \texttt{TOKEN\_RIGHT\_PAREN} \\
            \texttt{\{} & \texttt{TOKEN\_LEFT\_BRACE} \\
            \texttt{\}} & \texttt{TOKEN\_RIGHT\_BRACE} \\
            \texttt{,} & \texttt{TOKEN\_COMMA} \\
            \texttt{.} & \texttt{TOKEN\_DOT} \\
            \texttt{..} & \texttt{TOKEN\_DOUBLE\_DOT} \\
            \texttt{...} & \texttt{TOKEN\_TRIPLE\_DOT} \\
            \texttt{-} & \texttt{TOKEN\_MINUS} \\
		\end{tabular}
		\caption{}
	\end{subtable}
	\begin{subtable}{0.45\textwidth}
		\centering
        \begin{tabular}{ l l }
            \textbf{Pattern} & \textbf{Token} \\
            \texttt{+} & \texttt{TOKEN\_PLUS} \\
            \texttt{/} & \texttt{TOKEN\_SLASH} \\
            \texttt{*} & \texttt{TOKEN\_STAR} \\
            \texttt{->} & \texttt{TOKEN\_RIGHT\_ARROW} \\
            \texttt{!} & \texttt{TOKEN\_BANG} \\
            \texttt{!=} & \texttt{TOKEN\_BANG\_EQUAL} \\
            \texttt{=} & \texttt{TOKEN\_EQUAL} \\
            \texttt{==} & \texttt{TOKEN\_EQUAL\_EQUAL} \\
            \texttt{>} & \texttt{TOKEN\_HIGHER} \\
            \texttt{>=} & \texttt{TOKEN\_HIGHER\_EQUAL} \\
		\end{tabular}
		\caption{}
	\end{subtable}
	\caption{Nuua tokens (1)}
    \label{fig:nuua_tokens_1}
\end{table}

\begin{table}[H]
	\centering
	\begin{subtable}{0.45\textwidth}
		\centering
        \begin{tabular}{ l l }
            \textbf{Pattern} & \textbf{Token} \\
            \texttt{:} & \texttt{TOKEN\_COLON} \\
            \texttt{return} & \texttt{TOKEN\_RETURN} \\
            \texttt{print} & \texttt{TOKEN\_PRINT} \\
            \texttt{use} & \texttt{TOKEN\_USE} \\
		\end{tabular}
		\caption{}
	\end{subtable}
	\begin{subtable}{0.45\textwidth}
		\centering
        \begin{tabular}{ l l }
            \textbf{Pattern} & \textbf{Token} \\
            \texttt{from} & \texttt{TOKE\_FROM} \\
            \texttt{elif} & \texttt{TOKEN\_ELIF} \\
            \texttt{in} & \texttt{TOKEN\_IN} \\
            \texttt{export} & \texttt{TOKEN\_EXPORT} \\
		\end{tabular}
		\caption{}
	\end{subtable}
	\caption{Nuua tokens (2)}
    \label{fig:nuua_tokens_2}
\end{table}

\begin{table}[H]
	\centering
	\begin{subtable}{0.45\textwidth}
		\centering
        \begin{tabular}{ l l }
            \textbf{Pattern} & \textbf{Token} \\
            \texttt{<} & \texttt{TOKEN\_LOWER} \\
            \texttt{<=} & \texttt{TOKEN\_LOWER\_EQUAL} \\
            \texttt{<STRING\_EXPR>} & \texttt{TOKEN\_IDENTIFIER} \\
            \texttt{<INTEGER\_EXPR>} & \texttt{TOKEN\_STRING} \\
            \texttt{<FLOAT\_EXPR>} & \texttt{TOKEN\_FLOAT} \\
            \texttt{as} & \texttt{TOKEN\_AS} \\
            \texttt{or} & \texttt{TOKEN\_OR} \\
            \texttt{and} & \texttt{TOKEN\_AND} \\
            \texttt{class} & \texttt{TOKEN\_CLASS} \\
            \texttt{fun} & \texttt{TOKEN\_FUN} \\
		\end{tabular}
		\caption{}
	\end{subtable}
	\begin{subtable}{0.45\textwidth}
		\centering
        \begin{tabular}{ l l }
            \textbf{Pattern} & \textbf{Token} \\
            \texttt{else} & \texttt{TOKEN\_ELSE} \\
            \texttt{true} & \texttt{TOKEN\_TRUE} \\
            \texttt{false} & \texttt{TOKEN\_FALSE} \\
            \texttt{while} & \texttt{TOKEN\_WHILE} \\
            \texttt{for} & \texttt{TOKEN\_FOR} \\
            \texttt{if} & \texttt{TOKEN\_IF} \\
            \texttt{\textbackslash 0} & \texttt{TOKEN\_EOF} \\
            \texttt{[} & \texttt{TOKEN\_LEFT\_SQUARE} \\
            \texttt{]} & \texttt{TOKEN\_RIGHT\_SQUARE} \\
            \texttt{=>} & \texttt{TOKEN\_BIG\_RIGHT\_ARROW} \\
		\end{tabular}
		\caption{}
	\end{subtable}
	\caption{Nuua tokens (3)}
    \label{fig:nuua_tokens_3}
\end{table}

To make an efficient token representation, instead of copying the \texttt{char} array (lexeme) that matches that token pattern
a reference to the first character of the token is saved into the token instance and then the length is specified to know how many
chars it does consist of. That way, no unnecessary memory is added to each token. The only needed thing is to keep the source code \texttt{char}
array in memory and avoid reallocations till the token instance is not needed anymore. \autoref{fig:scanner_token} shows a visual representation.
\autoref{ls:token_class} shows the token representation that is used in the Nuua compiler.

Among other information, the type of the token (as shown in \autoref{fig:nuua_tokens_1}, \autoref{fig:nuua_tokens_2} and \autoref{fig:nuua_tokens_3})
is also stored as part of the token. Additionally, the line and column where that token was found is also stored.\\

\begin{figure}[H]
    \centering
    \begin{tikzpicture}[node distance=1cm]
        % Nodes
        \node[square, minimum height=0.75cm, minimum width=1cm] (n0) {Source code};
        \node[square, minimum height=0.75cm, minimum width=1cm, right = -0.025cm of n0.east] (n1) {...};
        \node[square, minimum height=0.75cm, minimum width=1cm, right of=n1] (n2) {\texttt{w}};
        \node[square, minimum height=0.75cm, minimum width=1cm, right of=n2] (n3) {\texttt{h}};
        \node[square, minimum height=0.75cm, minimum width=1cm, right of=n3] (n4) {\texttt{i}};
        \node[square, minimum height=0.75cm, minimum width=1cm, right of=n4] (n5) {\texttt{l}};
        \node[square, minimum height=0.75cm, minimum width=1cm, right of=n5] (n6) {\texttt{e}};
        \node[square, minimum height=0.75cm, minimum width=1cm, right of=n6] (n7) {...};
        % Token nodes
        \node[square, anchor=west, minimum height=0.75cm] at ($(n0.south west) + (0, -2)$) (token) {Token Instance};
        \node[square, minimum height=0.75cm, right = -0.025cm of token.east] (token1) {\texttt{const char *start = \phantom{X}}};
        \node[square, minimum height=0.75cm, right = -0.025cm of token1.east] (token2) {\texttt{const uint32\_t length = 5}};
        % Lines
        \draw[dot_arrow] ($(token1.east) + (-0.2, -0.05)$) -- (n2.south);
    \end{tikzpicture}
    \caption{Lexer token technique}
    \label{fig:scanner_token}
\end{figure}

\begin{listing}[H]
\begin{minted}{cpp}
class Token
{
    public:
        // The type of the token.
        const TokenType type;
        // A pointer to the first char.
        // (Not used as a null terminated char array)
        const char *start;
        // The length of the token.
        const uint32_t length;
        // The line where it is found.
        const line_t line;
        // The column where it is found.
        const column_t column;
        // String representation of the token names.
        static std::vector<std::string> token_names;
        // Pattern matching a token name.
        static std::vector<std::string> type_names;
        // Contains the escaped chars of the language.
        static const std::unordered_map<char, char> escaped_chars;
        // Create a new instance of a token given the
        // required parameters.
        Token(
            const TokenType type,
            const char *start,
            const uint32_t length,
            const line_t line,
            const column_t column
        ) : type(type), start(start), length(length),
            line(line), column(column) {}
        // Debug the token by printing it on the
        // screen with the correct format.
        void debug_token() const;
        // Convert the token to a string representation.
        std::string to_string() const;
        // Convert the token to its pattern string representation.
        std::string to_type_string() const;
};
\end{minted}
\caption{Token class}
\label{ls:token_class}
\end{listing}

\section{Lexer class}

The lexer class represents a lexer instance to scan a file and return a list of tokens.
\autoref{ls:lexer_class} shows the code used in the Nuua compiler to transform a \texttt{char} array
corresponding to the source file to a list of token instances.

\begin{listing}[H]
\begin{minted}{cpp}
class Lexer
{
    // Stores the file where the tokens are being scanned.
    std::shared_ptr<const std::string> file;
    // Stores the start char of the token being scanned.
    const char *start;
    // Stores the current char of the token beeing scanned.
    const char *current;
    // Stores the current line in the source file.
    line_t line;
    // Stores the current column in the source file.
    column_t column;
    // Stores a list of reserved words for the identifiers.
    static const std::unordered_map<std::string, TokenType> reserved_words;
    // Generates a token error.
    const std::string token_error() const;
    // Build the token and set the start char to the current one.
    Token make_token(TokenType type);
    // Helper to build tokens.
    bool match(const char c);
    TokenType is_string(bool simple);
    TokenType is_number();
    TokenType is_identifier();
    // Reads a file and stores the contents in the current instance.
    void read_from_file(const std::shared_ptr<const std::string> &file);
    public:
        // Stores the source code of the file.
        std::unique_ptr<std::string> source;
        // Scans the source and stores the tokens.
        void scan(std::unique_ptr<std::vector<Token>> &tokens);
        // Initializes a lexer given a file name.
        Lexer(const std::shared_ptr<const std::string> &file);
};
\end{minted}
\caption{Lexer class}
\label{ls:lexer_class}
\end{listing}
