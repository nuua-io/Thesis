As a result of the hard work involved in the development of this thesis, the final result satisfies all the primary objectives set at \autoref{sec:objectives}.
The final result gets close to other programming languages in terms of project structure but also in terms of performance according to some initial
tests done. However, benchmarking these systems is a very complex job and therefore it's not part of this thesis.

\section{Further Evolution}
\label{sec:forthcoming}

Although all the objects are complete there are still many things that could be implemented into the language, the compiler, and the interpreter.
Some of these things are very important and they are found in most mature programming languages. There are a set of features that are planned to be
implemented into the language, the compiler, and the interpreter.

Some of this features are not simple changes or additions, in fact, some of the features provided here can probably take months to implement,
therefore, they can't be included as part of this thesis and are just listed here for future reference.

\begin{itemize}
    \item A C foreign function interface (FFI) to call C code using Nuua. This should allow the use of libc and other existing C code.
    \item A propper I/O interface to read and write to files, including stdin, stdout, stderr and removing the print statement.
    \item Function overloading.
    \item Generic programming
    \item Extend and create a proper useful standard library.
    \item A website (\href{https://nuua.io}{nuua.io}) to announce, showcase and write all the documentation.
    \item A centralized package/module manager to automate the process of using external modules at a certain version.
    \item Use a wider string representation for the compiler. Change from \texttt{std::string} to something wider like
        \texttt{std::u16string} or \texttt{std::wstring}.
    \item Support more binary operators such as \texttt{\%, +=, -=, /= or *=}.
    \item Add more optimizations to the compiler.
\end{itemize}
