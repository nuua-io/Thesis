The job of the parser is to transform the list of tokens returned by the Lexer into an abstract syntax tree.\\

\begin{figure}[H]
    \centering
    \begin{tikzpicture}
        [node distance=1cm]

        % Nodes of the layered system
        \node[block] (input) {List of tokens};
        \node[block,right=of input] (parser) {Parser};
        \node[block,right=of lexer] (output) {Abstract Syntax Tree};

        % Lines
        \draw[line] (input) -- (parser);
        \draw[line] (parser) -- (output);

    \end{tikzpicture}

    % Caption and Label
    \caption{Parser overview}
    \label{fig:parser_overview}
\end{figure}

There are a lot of different algorithms and strategies for parsing different set of grammars.
There are also a lot of parser generators out there that can help building a parser. However, this thesis implements
a hand written top down recursive descend predictive parser for the following reasons:

\begin{enumerate}
    \item To have as much control as needed over the parser. Specially for error reporting.
    \item To enforce a zero-dependency policy.
    \item Easy to build and fast enough for the job.
    \item To learn how to build a recursive descend parser and learn how it works.
\end{enumerate}

Recursive descend parsers are probably the simplest way to build a parser as mentioned in \autocite[Section~6]{crafting_interpreters}.
In fact, even if it's a simple parser it's used in very famous and large projects like \texttt{GCC} or \texttt{V8}.

As can be seen on \autocite{top_down_parsing} the top down parers can be classified into the ones using backtracking and the ones using a predictive technique.

\begin{itemize}
    \item \emph{Backtracking parsers}: Backtracking parsers try to guess the correct production rule based on the current token. This guess can however, lead to a
        dead-end and therefore a backtrack is needed to go back and make anothe decision. This is hoever, very inefficient for programming languages due the ammount
        of nonterminals found in them.
    \item \emph{Predictive parsers}: Predictive parsers choose the production rule acording to the current token and the next token found. The next tooken is called
        a lookahead token. Predictive parsers have a major issue with left-recursion and therefore, the language grammar needs to be adapted to it.
        The Nuua grammar as seen on \autoref{sec:nuua_grammar} is already adapted for left-recursion based on the solution mentioned on
        \autocite[Section~6]{crafting_interpreters}.
\end{itemize}

A recursive descend parser is a possible technique for implementing a top-down predictive parser that consists in the creation of a funciton for each
nonterminal found in the grammar and then calling the functions depending on the current token and the lookahead.

\section{Abstract Syntax Tree}

The abstract syntax tree (AST) is the data structure used to represent the program in memory. This tree contains the
representation of the program starting from an initial node \texttt{program} as seen in \autoref{sec:program_tld}.
\autoref{fig:ast_example} shows an example AST.

\begin{figure}[H]
	\centering
	\begin{subtable}{0.45\textwidth}
		\centering
        \texttt{1 - 2 * -3}
		\caption{Program}
	\end{subtable}
	\begin{subtable}{0.45\textwidth}
		\centering
        \texttt{TOKEN\_INTEGER TOKEN\_MINUS TOKEN\_INTEGER TOKEN\_STAR TOKEN\_MINUS TOKEN\_INTEGER}
		\caption{List of tokens}
	\end{subtable}
    \begin{subfigure}{0.45\textwidth}
		\centering
        \begin{tikzpicture}[node distance=1.75cm]
            % Nodes
            \node[state] (1) {\texttt{-}};
            \node[state, below left of=1] (2) {\texttt{1}};
            \node[state, below right of=1] (3) {\texttt{*}};
            \node[state, below left of=3] (4) {\texttt{2}};
            \node[state, below right of=3] (5) {\texttt{-}};
            \node[state, below of=5] (6) {\texttt{3}};
            % Lines
            \draw (1) -- (2);
            \draw (1) -- (3);
            \draw (3) -- (4);
            \draw (3) -- (5);
            \draw (5) -- (6);
        \end{tikzpicture}
		\caption{Abstract syntax tree}
	\end{subfigure}
    \begin{subfigure}{0.45\textwidth}
		\centering
        \begin{tikzpicture}[node distance=1.75cm]
            % Nodes
            \node[state] (0) {\texttt{P(1)}};
            \node[state, below of=0] (1) {\texttt{B(+)}};
            \node[state, below left of=1] (2) {\texttt{1}};
            \node[state, below right of=1] (3) {\texttt{B(*)}};
            \node[state, below left of=3] (4) {\texttt{2}};
            \node[state, below right of=3] (5) {\texttt{U(-)}};
            \node[state, below of=5] (6) {\texttt{3}};
            % Legend
            \node[text width=4.5cm, align=left, right = 0.5cm of 0.east] (l) {

                { \footnotesize \noindent
                    \texttt{P(n):\\\quad Program (n statements)}\\\texttt{B(operation):\\\quad Binary operation}\\\texttt{U(operation):\\\quad Unary operation}
                }
            };
            % Lines
            \draw (0) -- (1);
            \draw (1) -- (2);
            \draw (1) -- (3);
            \draw (3) -- (4);
            \draw (3) -- (5);
            \draw (5) -- (6);
        \end{tikzpicture}
		\caption{Nuua AST}
	\end{subfigure}
	\caption{Example abstract syntax tree}
    \label{fig:ast_example}
\end{figure}

\section{Data Types}

The parser layer is responsible for creating the type class used in the nodes and in the blocks to determine the type of the values.
Since some nodes will store information about their type and it will be added when the semantic analyzer runs. The blocks also need
the type of the variables stores so it's convinient to define it here.

The data types supported in nuua are already explained in \autoref{sec:nuua_data_types}, therefore \autoref{ls:nuua_data_types} shows the
available data types supported with the addition of a value type that contains no type. This is used in call expression to functions that don't have
a return type. This type is only used by the compiler and not the interpreter.

\begin{listing}[H]
\begin{minted}{cpp}
// Determines the available native types in nuua.
typedef enum : uint8_t {
    VALUE_INT, VALUE_FLOAT, VALUE_BOOL,
    VALUE_STRING, VALUE_LIST, VALUE_DICT, VALUE_FUN,
    VALUE_OBJECT, VALUE_NO_TYPE
} ValueType;
\end{minted}
\caption{Nuua data types}
\label{ls:nuua_data_types}
\end{listing}

The type class is very complex. In fact, it must store the \texttt{VALUE\_TYPE} and additional information about the type, for example
the inner type for values like lists or dictionaries. For the function type it must store the function parameter types and its return type. For class types
it must store the class name used. \autoref{ls:type_class} shows the type class used in the Nuua compiler. As it can be seen, there are lot
of additional methods bound to it to help working with types. Those include copy methods, reset methods, type comparisons and additionally,
methods that help dealing with the long list of casts, unary and binary operations found in \autoref{sec:expressions}.

\begin{listing}[H]
\begin{minted}{cpp}
class Type
{
    // Stores the asociation between the types
    // as string and the value type.
    static const std::unordered_map<std::string, ValueType> value_types;
    // Determines the string representation
    // of the value types.
    static const std::vector<std::string> types_string;
    public:
        // Stores the type.
        ValueType type;
        // Stores the inner type if needed.
        // Used as return type for functions.
        std::shared_ptr<Type> inner_type;
        // Class name.
        std::string class_name;
        // Store the parameters of a function type.
        std::vector<std::shared_ptr<Type>> parameters;
        // Create a value type given the type.
        Type() : type(VALUE_NO_TYPE) {}
        Type(const ValueType type)
            : type(type) {}
        // Create a type given a value type and the inner type.
        Type(
            const ValueType type,
            const std::shared_ptr<Type> &inner_type
        ) : type(type), inner_type(inner_type) {}
        // Create a type given a string representation of it.
        Type(const std::string &name);
        // Create a type given an expression and a
        // list of code blocks to know the variable values.
        Type(
            const std::shared_ptr<Expression> &rule,
            const std::vector<std::shared_ptr<Block>> *blocks
        );
        // Create a function type given the function itself.
        Type(const std::shared_ptr<Function> &fun);
        // Additional methods used as helpers for casts,
        // unary and binary operations and for other operations
        // like copy or
        // ...
};
\end{minted}
\caption{Type class}
\label{ls:type_class}
\end{listing}

\section{Block Scope and symbol table}

The parser layer will also create the block representation explained in \autoref{sec:nuua_scopes}. The way the scope block is built must be
explained here since it will be attached as part of some of the AST nodes. This acts as a symbol table for the specific scope.

A symbol table is used to store the name of the variables in the block and further verify if they are declared or get specific data of them.
For example, the symbol table can verify if a variable is declared, the type of the variable, the register where it is placed, etc.
Aditionally, the block does also store the defined classes. This is done due the fact that the block class is also used as the module scope internaly.

\subsection{Block Variable Type}

The block variable type is the representation of a variable stores in the block. A variable must include the following fields:

\begin{itemize}
    \item The variable type.
    \item The AST node where it appears (The declaration).
    \item A register where this variable is stored.
    \item If the variable is exported or not.
    \item The node where it was last used.
\end{itemize}

The variable name however, is not part of this class, since to reference this variable class, a hashmap is used and the hashmap key is variable name.

\autoref{ls:block_variable_class} shows the class representation of a block variable type.

\begin{listing}[H]
\begin{minted}{cpp}
class BlockVariableType
{
    public:
        // Represents the variable type.
        std::shared_ptr<Type> type;
        // Stores the AST node where this variable is.
        std::shared_ptr<Node> node;
        // Represents the register where it's stored.
        register_t reg = 0;
        // Determines in the variable is exported.
        // (only applies to TLDs).
        bool exported = false;
        // Represents the last use of the variable (Variable life).
        std::shared_ptr<Node> last_use;
        // Constructors.
        BlockVariableType() {};
        BlockVariableType(
            const std::shared_ptr<Type> &type,
            const std::shared_ptr<Node> &node,
            const bool exported = false
        ) : type(type), node(node), exported(exported) {}
};
\end{minted}
\caption{BlockVariableType class}
\label{ls:block_variable_class}
\end{listing}

\subsection{Block Class Type}

The block class type is similar to the block variable type but stores much less fields. Also this information is only used in the module scope.
The block class type stores the following information:

\begin{itemize}
    \item The block scope of the class.
    \item If the class is exported or not.
    \item The node where it was last used.
\end{itemize}

\autoref{ls:block_class_class} shows the class representation of the block class type. It can be seen that a forward declaration to
the block class must be used till its defined.

\begin{listing}[H]
\begin{minted}{cpp}
class Block; // Forward declaration.
class BlockClassType
{
    public:
        // Stores the block scope of the class.
        std::shared_ptr<Block> block;
        // Determines if the class is exported.
        bool exported = false;
        // Stores the AST node where this variable is.
        std::shared_ptr<Node> node;
        // Constructors.
        BlockClassType() {}
        BlockClassType(
            const std::shared_ptr<Block> &block,
            const std::shared_ptr<Node> &node,
            const bool exported = false
        ) : block(block), node(node), exported(exported) {}
};
\end{minted}
\caption{BlockClassType class}
\label{ls:block_class_class}
\end{listing}

\subsection{Block Class}

With the form of the variables and the classes that are stored in the block the final definition for a block can be given, therefore, the block must
then be able to store variables, classes and additionally provide with methods to help perform different opterations on them.

\autoref{ls:block_class} shows the block class used in the Nuua compiler. As it can be seen, two hashmaps are used to store the relation between
variable names and its class shown in \autoref{ls:block_variable_class} and the same with the class name with its class shown in \autoref{ls:block_class_class}.

\begin{listing}[H]
\begin{minted}{cpp}
class Block
{
    public:
        // Stores the variable name and the type of it.
        std::unordered_map<std::string, BlockVariableType> variables;
        // Stores the custom types of the block.
        std::unordered_map<std::string, BlockClassType> classes;
        // Gets a variable from the current block or returns nullptr.
        BlockVariableType *get_variable(const std::string &name);
        // Gets a class from the current block or returns nullptr.
        BlockClassType *get_class(const std::string &name);
        // Sets a variable.
        void set_variable(const std::string &name, const BlockVariableType &var);
        // Sets a class.
        void set_class(const std::string &name, const BlockClassType &c);
        // Determines if a variable is exported.
        bool is_exported(const std::string &name);
        // Determines if a class is exported.
        bool is_exported_class(const std::string &name);
        // Determines if the block have a variable.
        bool has(const std::string &name);
        // Determines if the block have a class.
        bool has_class(const std::string &name);
        // Debug the block by printing it to the screen.
        void debug() const;
        // Helper to get a single variable out of a list of blocks.
        // It iterates through it starting from the end till the front.
        static BlockVariableType *get_single_variable(const std::string &name, const std::vector<std::shared_ptr<Block>> *blocks);
};
\end{minted}
\caption{Block class}
\label{ls:block_class}
\end{listing}

\section{Tree Nodes}
\label{sec:tree_nodes}

A node in the AST is represented by the class found \autoref{ls:node_class}

\begin{listing}[H]
\begin{minted}{cpp}
class Node
{
    public:
        // Stores the real rule of the node.
        const Rule rule;
        // Stores the file where it's found.
        std::shared_ptr<const std::string> file;
        // Stores the line where it's found.
        line_t line;
        // Stores the column where it's found.
        column_t column;
        // Constructor.
        Node(
            const Rule r,
            const std::shared_ptr<const std::string> &f,
            const line_t l, const column_t c
        ) : rule(r), file(f), line(l), column(c) {};
};
\end{minted}
\caption{Node class}
\label{ls:node_class}
\end{listing}

However, nodes are categorized into two sections. Nodes can be either statements or expressions. Therefore, two classes
are used as nodes instead as shown in \autoref{ls:expr_stmt_classes}.

\begin{listing}[H]
\begin{minted}{cpp}
class Expression : public Node
{
    public:
        explicit Expression(const Node &node)
            : Node(node) {};
};
class Statement : public Node
{
    public:
        explicit Statement(const Node &node)
            : Node(node) {};
};
\end{minted}
\caption{Expression and Statement classes}
\label{ls:expr_stmt_classes}
\end{listing}

This two classes extend the node class so they inherit all the attributes from it.

To work with all the nodes in a more effective way, the definition in \autoref{ls:node_def} is added.

\begin{listing}[H]
\begin{minted}{cpp}
#define NODE_PROPS std::shared_ptr<const std::string> &file, const line_t line, const column_t column
\end{minted}
\caption{Node properties definition}
\label{ls:node_def}
\end{listing}

\subsection{Integer}

The integer node only needs the value that corresponds to it. In this case a C++ \texttt{int64\_t} type is used as shown in
\autoref{ls:integer_node}.

\begin{listing}[H]
\begin{minted}{cpp}
class Integer : public Expression
{
    public:
        int64_t value;
        Integer(NODE_PROPS, const int64_t v)
            : Expression({ RULE_INTEGER, file, line, column }),
              value(v) {};
};
\end{minted}
\caption{Integer Node}
\label{ls:integer_node}
\end{listing}

\subsection{Float}

The float node only needs the value that corresponds to it. In this case a C++ \texttt{double} type is used as shown in
\autoref{ls:float_node}.

\begin{listing}[H]
\begin{minted}{cpp}
class Float : public Expression
{
    public:
        double value;
        Float(NODE_PROPS, const double v)
            : Expression({ RULE_FLOAT, file, line, column }),
              value(v) {};
};
\end{minted}
\caption{Float Node}
\label{ls:float_node}
\end{listing}

\subsection{String}

The string node only needs the value that corresponds to it. In this case a C++ \texttt{std::string} type is used as shown in
\autoref{ls:string_node}.

\begin{listing}[H]
\begin{minted}{cpp}
class String : public Expression
{
    public:
        std::string value;
        String(NODE_PROPS, const std::string &v)
            : Expression({ RULE_STRING, file, line, column }),
              value(v) {};
};
\end{minted}
\caption{String Node}
\label{ls:string_node}
\end{listing}

\subsection{Boolean}

The boolean node only needs the value that corresponds to it. In this case a C++ \texttt{bool} type is used as shown in
\autoref{ls:bool_node}.

\begin{listing}[H]
\begin{minted}{cpp}
class Boolean : public Expression
{
    public:
        bool value;
        Boolean(NODE_PROPS, const bool v)
            : Expression({ RULE_BOOLEAN, file, line, column }),
              value(v) {};
};
\end{minted}
\caption{Boolean Node}
\label{ls:bool_node}
\end{listing}

\subsection{List}

The list node needs the value that corresponds to it and a type parameter that's filled in by the semantic analyzer.
In this case a C++ \texttt{std::vector<std::shared\_ptr<Expression>>} type is used as shown in \autoref{ls:list_node}.
The vector contains other expressions found inside the list.

\begin{listing}[H]
\begin{minted}{cpp}
class List : public Expression
{
    public:
        std::vector<std::shared_ptr<Expression>> value;
        // Stores the list type since it's complex to analyze later.
        std::shared_ptr<Type> type;
        List(
            NODE_PROPS,
            const std::vector<std::shared_ptr<Expression>> &v
        ) : Expression({ RULE_LIST, file, line, column }),
            value(v) {};
};
\end{minted}
\caption{List Node}
\label{ls:list_node}
\end{listing}

\subsection{Dictionary}

The dictionary node needs the key value pairs of strings and expressions and the key order. In this case a C++ \texttt{std::unordered\_map<std::string, std::shared\_ptr<Expression>>} type is used as shown in \autoref{ls:dict_node}. The type is also added and used by the semantic analyzer.

\begin{listing}[H]
\begin{minted}{cpp}
class Dictionary : public Expression
{
    public:
        std::unordered_map<std::string, std::shared_ptr<Expression>> value;
        std::vector<std::string> key_order;
        // Stores the dict type since it's complex to analyze later.
        std::shared_ptr<Type> type;
        Dictionary(
            NODE_PROPS,
            const std::unordered_map<std::string, std::shared_ptr<Expression>> &v,
            const std::vector<std::string> &ko
        ) : Expression({ RULE_DICTIONARY, file, line, column }),
            value(std::move(v)),
            key_order(std::move(ko)) {};
};
\end{minted}
\caption{Dictionary Node}
\label{ls:dict_node}
\end{listing}

\subsection{Group}

The group node consists of a single expression as seen in \autoref{ls:group_node}.

\begin{listing}[H]
\begin{minted}{cpp}
class Group : public Expression
{
    public:
        std::shared_ptr<Expression> expression;
        Group(NODE_PROPS, const std::shared_ptr<Expression> &v)
            : Expression({ RULE_GROUP, file, line, column }),
              expression(std::move(v)) {};
};
\end{minted}
\caption{Group Node}
\label{ls:group_node}
\end{listing}

\subsection{Unary}

The Unary node consists of the token operation to do and the right expression. It also stores the type of unary operation as seen in
\autoref{sec:unary_expression}. \autoref{ls:unary_node} shows the representation of the unary node.

\begin{listing}[H]
\begin{minted}{cpp}
class Unary : public Expression
{
    public:
        Token op;
        std::shared_ptr<Expression> right;
        // Determines what type of unary operation
        // will be performed, no need to store a Type.
        UnaryType type = (UnaryType) NULL;
        Unary(NODE_PROPS, const Token &o, const std::shared_ptr<Expression> &r)
            : Expression({ RULE_UNARY, file, line, column }),
              op(o),
              right(std::move(r)) {};
};
\end{minted}
\caption{Unary Node}
\label{ls:unary_node}
\end{listing}

\subsection{Binary}

The Binary node consists of the token operation to do and the left and right expression. It also stores the type of binary operation as seen in
\autoref{sec:binary_expression}. \autoref{ls:binary_node} shows the representation of the binary node.

\begin{listing}[H]
\begin{minted}{cpp}
class Binary : public Expression
{
    public:
        std::shared_ptr<Expression> left;
        Token op;
        std::shared_ptr<Expression> right;
        // Determines what type of binary operation will be performed.
        BinaryType type = (BinaryType) NULL;
        Binary(
            NODE_PROPS,
            const std::shared_ptr<Expression> &l,
            const Token &o,
            const std::shared_ptr<Expression> &r
        ) : Expression({ RULE_BINARY, file, line, column }),
            left(std::move(l)),
            op(o),
            right(std::move(r)) {};
};
\end{minted}
\caption{Binary Node}
\label{ls:binary_node}
\end{listing}

\subsection{Variable}

The Variable node consists of the the name of the variable. \autoref{ls:variable_node} shows the representation of the variable node.

\begin{listing}[H]
\begin{minted}{cpp}
class Variable : public Expression
{
    public:
        std::string name;
        Variable(NODE_PROPS, const std::string &n)
            : Expression({ RULE_VARIABLE, file, line, column }),
              name(n) {};
};
\end{minted}
\caption{Variable Node}
\label{ls:variable_node}
\end{listing}

\subsection{Assign}

The Assign node consists of the target (the expression to be assigned) and the value of the assignment.
\autoref{ls:assign_node} shows the representation of the assign node. It also stores a boolean that determines if
the assignment is done to an access target (an inner value of a Nuua iterator). This boolean is used by the semantic analyzer.

\begin{listing}[H]
\begin{minted}{cpp}
class Assign : public Expression
{
    public:
        std::shared_ptr<Expression> target;
        std::shared_ptr<Expression> value;
        bool is_access = false;
        Assign(
            NODE_PROPS,
            const std::shared_ptr<Expression> &t,
            const std::shared_ptr<Expression> &v
        ) : Expression({ RULE_ASSIGN, file, line, column }),
            target(std::move(t)),
            value(std::move(v)) {};
};
\end{minted}
\caption{Assign Node}
\label{ls:assign_node}
\end{listing}

\subsection{Logical}

The Logical node consists the logical operation (\texttt{or} or \texttt{and}) and the left and right expressions.
\autoref{ls:logical_node} shows the representation of the logical node.

\begin{listing}[H]
\begin{minted}{cpp}
class Logical : public Expression
{
    public:
        std::shared_ptr<Expression> left;
        Token op;
        std::shared_ptr<Expression> right;
        Logical(
            NODE_PROPS,
            const std::shared_ptr<Expression> &l,
            const Token &o,
            const std::shared_ptr<Expression> &r
        ) : Expression({ RULE_LOGICAL, file, line, column }),
            left(std::move(l)),
            op(o),
            right(std::move(r)) {};
};
\end{minted}
\caption{Logical Node}
\label{ls:logical_node}
\end{listing}

\subsection{Call}

The Call node consists the calle target and of the arguments provided to perform the call.
\autoref{ls:call_node} shows the representation of the call node. It also stores if the call has a return
depending on the callee. This helps the semantic analyzer determine where and when this call can be used since
this expression may have no value.

\begin{listing}[H]
\begin{minted}{cpp}
class Call : public Expression
{
    public:
        std::shared_ptr<Expression> target;
        std::vector<std::shared_ptr<Expression>> arguments;
        // Determines if the call target returns a value or not.
        bool has_return = false;
        Call(
            NODE_PROPS,
            const std::shared_ptr<Expression> &t,
            const std::vector<std::shared_ptr<Expression>> &a
        ) : Expression({ RULE_CALL, file, line, column }),
            target(std::move(t)),
            arguments(a) {};
};
\end{minted}
\caption{Call Node}
\label{ls:call_node}
\end{listing}

\subsection{Access}

The Access node consists the target to access and the index for it.
\autoref{ls:access_node} shows the representation of the access node.
The type represents the type of access (the type of iterator to access).

\begin{listing}[H]
\begin{minted}{cpp}
class Access : public Expression
{
    public:
        std::shared_ptr<Expression> target;
        std::shared_ptr<Expression> index;
        AccessType type = (AccessType) NULL;
        Access(
            NODE_PROPS,
            const std::shared_ptr<Expression> &t,
            const std::shared_ptr<Expression> &i
        ) : Expression({ RULE_ACCESS, file, line, column }),
            target(std::move(t)),
            index(std::move(i)) {};
};
\end{minted}
\caption{Access Node}
\label{ls:access_node}
\end{listing}

\subsection{Cast}

The Cast node consists the expression to cast and the type to cast it to.
\autoref{ls:cast_node} shows the representation of the access node.
The type of cast as show in \autoref{sec:cast_expression} is also stored and further used by the analyzer.

\begin{listing}[H]
\begin{minted}{cpp}
class Cast : public Expression
{
    public:
        std::shared_ptr<Expression> expression;
        std::shared_ptr<Type> type;
        CastType cast_type = (CastType) NULL;
        Cast(
            NODE_PROPS,
            const std::shared_ptr<Expression> &e,
            std::shared_ptr<Type> &t
        ) : Expression({ RULE_CAST, file, line, column }),
            expression(std::move(e)),
            type(std::move(t)) {}
};
\end{minted}
\caption{Cast Node}
\label{ls:cast_node}
\end{listing}

\subsection{Slice}

The Slice node consists the expressions of the target and the expressions of the start, end and step indexes.
\autoref{ls:slice_node} shows the representation of the access node.
The additional \texttt{is\_list} represents if the slice is done to a list or a string and it's used by the semantic analyzer.

\begin{listing}[H]
\begin{minted}{cpp}
class Slice : public Expression
{
    public:
        std::shared_ptr<Expression> target;
        std::shared_ptr<Expression> start;
        std::shared_ptr<Expression> end;
        std::shared_ptr<Expression> step;
        // Determines if it's a list or a string, used by Analyzer.
        bool is_list = false;
        Slice(
            NODE_PROPS,
            const std::shared_ptr<Expression> &t,
            const std::shared_ptr<Expression> &s,
            const std::shared_ptr<Expression> &e,
            const std::shared_ptr<Expression> &st
        ) : Expression({ RULE_SLICE, file, line, column }),
            target(std::move(t)),
            start(std::move(s)),
            end(std::move(e)),
            step(std::move(st)) {}
};
\end{minted}
\caption{Slice Node}
\label{ls:slice_node}
\end{listing}

\subsection{Range}

The Range node consists the expressions of the start and end indexes.
\autoref{ls:range_node} shows the representation of the access node.
The inclusive fields indicates if the range is inclusive or exclusive.

\begin{listing}[H]
\begin{minted}{cpp}
class Range : public Expression
{
    public:
        std::shared_ptr<Expression> start;
        std::shared_ptr<Expression> end;
        bool inclusive;
        Range(
            NODE_PROPS,
            const std::shared_ptr<Expression> &s,
            const std::shared_ptr<Expression> &e,
            const bool i
        ) : Expression({ RULE_RANGE, file, line, column }),
            start(std::move(s)),
            end(std::move(e)),
            inclusive(i) {}
};
\end{minted}
\caption{Range Node}
\label{ls:range_node}
\end{listing}

\subsection{Delete}

The Delete node consists the target expression to delete.
\autoref{ls:range_node} shows the representation of the access node.

\begin{listing}[H]
\begin{minted}{cpp}
class Delete : public Expression
{
    public:
        std::shared_ptr<Expression> target;
        Delete(
            NODE_PROPS,
            const std::shared_ptr<Expression> &t
        ) : Expression({ RULE_DELETE, file, line, column }),
            target(std::move(t)) {}
};
\end{minted}
\caption{Delete Node}
\label{ls:delete_node}
\end{listing}

\subsection{FunctionValue}

The FunctionValue node represents a value to build a function. The node must know the function name,
the parameters of it, the return type and the body. Aditionally the scope block is also stored for further use.
\autoref{ls:function_value_node} shows the representation of the FunctionValue node.

\begin{listing}[H]
\begin{minted}{cpp}
class FunctionValue : public Expression
{
    public:
        std::string name;
        std::vector<std::shared_ptr<Declaration>> parameters;
        std::shared_ptr<Type> return_type;
        std::vector<std::shared_ptr<Statement>> body;
        std::shared_ptr<Block> block;
        FunctionValue(
            NODE_PROPS,
            const std::string &n,
            const std::vector<std::shared_ptr<Declaration>> &p,
            std::shared_ptr<Type> &rt,
            const std::vector<std::shared_ptr<Statement>> &b
        ) : Expression({ RULE_FUNCTION, file, line, column }),
            name(n), parameters(p),
            return_type(std::move(rt)),
            body(b) {}
};
\end{minted}
\caption{FunctionValue Node}
\label{ls:function_value_node}
\end{listing}

\subsection{Object}

The Object node represents an object expression. The required informations are the class name and the arguments to initialize the instance (as a key value pair).
Aditionally the scope block is also stored for further use.
\autoref{ls:object_node} shows the representation of the Object node.

\begin{listing}[H]
\begin{minted}{cpp}
class Object : public Expression
{
    public:
        std::string name;
        std::unordered_map<std::string, std::shared_ptr<Expression>> arguments;
        Object(
            NODE_PROPS,
            const std::string &n,
            const std::unordered_map<std::string, std::shared_ptr<Expression>> &a
        ) : Expression({ RULE_OBJECT, file, line, column }),
            name(n),
            arguments(a) {}
};
\end{minted}
\caption{Object Node}
\label{ls:object_node}
\end{listing}

\subsection{Property}

The Property node represents an access to an object property.
The information needed is the expression to access and the name of the property.
\autoref{ls:property_node} shows the representation of the Property node.

\begin{listing}[H]
\begin{minted}{cpp}
class Property : public Expression
{
    public:
        std::shared_ptr<Expression> object;
        std::string name;
        Property(
            NODE_PROPS,
            const std::shared_ptr<Expression> &o,
            const std::string &n
        ) : Expression({ RULE_PROPERTY, file, line, column }),
            object(o),
            name(n) {}
};
\end{minted}
\caption{Property Node}
\label{ls:property_node}
\end{listing}

\subsection{Print}

The Print node only requires the expression to print.
\autoref{ls:print_node} shows the representation of the Print node.

\begin{listing}[H]
\begin{minted}{cpp}
class Print : public Statement
{
    public:
        std::shared_ptr<Expression> expression;
        Print(
            NODE_PROPS,
            const std::shared_ptr<Expression> &e
        ) : Statement({ RULE_PRINT, file, line, column }),
            expression(std::move(e)) {}
};
\end{minted}
\caption{Print Node}
\label{ls:print_node}
\end{listing}

\subsection{ExpressionStatement}

The ExpressionStatement node is used as a wrapper to treat an expression as a valid statement.
\autoref{ls:expression_statement_node} shows the representation of the ExpressionStatement node.

\begin{listing}[H]
\begin{minted}{cpp}
class ExpressionStatement : public Statement
{
    public:
        std::shared_ptr<Expression> expression;
        ExpressionStatement(
            NODE_PROPS,
            const std::shared_ptr<Expression> &e
        ) : Statement({ RULE_EXPRESSION_STATEMENT, file, line, column }),
            expression(std::move(e)) {}
};
\end{minted}
\caption{ExpressionStatement Node}
\label{ls:expression_statement_node}
\end{listing}

\subsection{Declaration}

The Declaration node is requires the name of the variable to declare and then a combination of the type and the initializer.
The type may be empty if the initializer is set, and the initializer may be empty if the type is set. Both of them may also be set
but not empty.
\autoref{ls:declaration_node} shows the representation of the Declaration node.

\begin{listing}[H]
\begin{minted}{cpp}
class Declaration : public Statement
{
    public:
        std::string name;
        std::shared_ptr<Type> type;
        std::shared_ptr<Expression> initializer;
        Declaration(
            NODE_PROPS,
            const std::string &n, std::shared_ptr<Type> &t,
            const std::shared_ptr<Expression> &i
        ) : Statement({ RULE_DECLARATION, file, line, column }),
            name(n),
            type(std::move(t)),
            initializer(std::move(i)) {};
};
\end{minted}
\caption{Declaration Node}
\label{ls:declaration_node}
\end{listing}

\subsection{Return}

The Return node does not need an expression to be valid, but one may be provided.
\autoref{ls:return_node} shows the representation of the Return node.

\begin{listing}[H]
\begin{minted}{cpp}
class Return : public Statement
{
    public:
        std::shared_ptr<Expression> value;
        Return(
            NODE_PROPS,
            const std::shared_ptr<Expression> &v = std::shared_ptr<Expression>()
        ) : Statement({ RULE_RETURN, file, line, column }),
            value(std::move(v)) {}
};
\end{minted}
\caption{Return Node}
\label{ls:return_node}
\end{listing}

\subsection{If}

The If node does need the condition expression and the then branch followed by the then block.
Aditionally, it may have an else branch and an else scope block.
\autoref{ls:if_node} shows the representation of the If node.

\begin{listing}[H]
\begin{minted}{cpp}
class If : public Statement
{
    public:
        std::shared_ptr<Expression> condition;
        std::vector<std::shared_ptr<Statement>> then_branch;
        std::vector<std::shared_ptr<Statement>> else_branch;
        std::shared_ptr<Block> then_block, else_block;
        If(
            NODE_PROPS,
            const std::shared_ptr<Expression> &c,
            const std::vector<std::shared_ptr<Statement>> &tb,
            const std::vector<std::shared_ptr<Statement>> &eb
        ) : Statement({ RULE_IF, file, line, column }),
            condition(std::move(c)),
            then_branch(tb),
            else_branch(eb) {};
};
\end{minted}
\caption{If Node}
\label{ls:if_node}
\end{listing}

\subsection{While}

The While node does need the condition expression and the body of it.
It also stores the scope block for further use.
\autoref{ls:while_node} shows the representation of the While node.

\begin{listing}[H]
\begin{minted}{cpp}
class While : public Statement
{
    public:
        std::shared_ptr<Expression> condition;
        std::vector<std::shared_ptr<Statement>> body;
        std::shared_ptr<Block> block;
        While(
            NODE_PROPS,
            const std::shared_ptr<Expression> &c,
            const std::vector<std::shared_ptr<Statement>> &b
        ) : Statement({ RULE_WHILE, file, line, column }),
            condition(std::move(c)),
            body(b) {};
};
\end{minted}
\caption{While Node}
\label{ls:while_node}
\end{listing}

\subsection{For}

The For node does require the iterator and the variable to store the loop value.
Aditionally it may have the index in case it's needed as part of the loop.
It also stores the scope block for further use.
\autoref{ls:for_node} shows the representation of the For node.

\begin{listing}[H]
\begin{minted}{cpp}
class For : public Statement
{
    public:
        std::string variable;
        std::string index;
        std::shared_ptr<Expression> iterator;
        std::vector<std::shared_ptr<Statement>> body;
        std::shared_ptr<Block> block;
        // Stores the type of the iterator.
        std::shared_ptr<Type> type;
        For(
            NODE_PROPS,
            const std::string &v,
            const std::string &i,
            const std::shared_ptr<Expression> &it,
            const std::vector<std::shared_ptr<Statement>> &b
        ) : Statement({ RULE_FOR, file, line, column }),
            variable(v),
            index(i),
            iterator(std::move(it)), body(b) {}
};
\end{minted}
\caption{For Node}
\label{ls:for_node}
\end{listing}

\subsection{Function}

The Function node wraps the \texttt{FunctionValue} node as a statement.
\autoref{ls:function_node} shows the representation of the Function node.

\begin{listing}[H]
\begin{minted}{cpp}
class Function : public Statement
{
    public:
        std::shared_ptr<FunctionValue> value;
        Function(const std::shared_ptr<FunctionValue> &v)
            : Statement({ RULE_FUNCTION, v->file, v->line, v->column }),
                value(v) {}
};
\end{minted}
\caption{Function Node}
\label{ls:function_node}
\end{listing}

\subsection{Use}

The Use node requires the module name to import and an optional target list. Targets can be empty and all exported targets
will be imported. It also requires a property to store the code of the module (the AST of that particular module) and
the module scope block of it.
\autoref{ls:use_node} shows the representation of the Use node.

\begin{listing}[H]
\begin{minted}{cpp}
class Use : public Statement
{
    public:
        std::vector<std::string> targets;
        std::shared_ptr<const std::string> module;
        std::shared_ptr<std::vector<std::shared_ptr<Statement>>> code;
        std::shared_ptr<Block> block;
        Use(
            NODE_PROPS,
            const std::vector<std::string> &t,
            const std::shared_ptr<const std::string> &m
        ) : Statement({ RULE_USE, file, line, column }),
            targets(t),
            module(std::move(m)) {};
};
\end{minted}
\caption{Use Node}
\label{ls:use_node}
\end{listing}

\subsection{Export}

The Export node just requires the statement that is exporting.
\autoref{ls:export_node} shows the representation of the Export node.

\begin{listing}[H]
\begin{minted}{cpp}
class Export : public Statement
{
    public:
        std::shared_ptr<Statement> statement;
        Export(NODE_PROPS, const std::shared_ptr<Statement> &s)
            : Statement({ RULE_EXPORT, file, line, column }),
                statement(std::move(s)) {}
};
\end{minted}
\caption{Export Node}
\label{ls:export_node}
\end{listing}

\subsection{Class}

The Class node needs the class name, the body of the class and it saves the scope block of the class for further use.
\autoref{ls:class_node} shows the representation of the Class node.

\begin{listing}[H]
\begin{minted}{cpp}
class Class : public Statement
{
    public:
        std::string name;
        std::vector<std::shared_ptr<Statement>> body;
        std::shared_ptr<Block> block;
        Class(
            NODE_PROPS,
            const std::string &n,
            const std::vector<std::shared_ptr<Statement>> &b
        ) : Statement({ RULE_CLASS, file, line, column }),
            name(n),
            body(b) {}
};
\end{minted}
\caption{Class Node}
\label{ls:class_node}
\end{listing}

\section{Parser Class}

The parser class is able to parse the program tree by using some recursive logic. When a use declaration is found when parsing,
the contents of the imported module are parsed using a new instance of the parser class. That way, the first parser class contains
the AST of all the program. It's also good to note that the parser class formats the given paths to absolute paths and checks for the
file location to determine if the imported module belongs to a relative file or the standard library. The parser class uses a similar technique
as the Lexer when dealing with the current token. The parser have a pointer to the current token beeing parser and the lookahead token can be
calculated by using pointer arithmentic. \autoref{ls:parser_class} shows the implementation of the parser.

\begin{listing}[H]
\begin{minted}{cpp}
class Parser
{
    // Stores the current parsing file.
    std::shared_ptr<const std::string> file;
    // Stores a pointer to the current token beeing parsed.
    Token *current = nullptr;
    // Consumes a token and returns it for futher use.
    Token *consume(const TokenType type, const std::string &message);
    // Returns true if the token type matches the current token.
    bool match(const TokenType token);
    // Returns true if any of the given token types matches the current token.
    bool match_any(const std::vector<TokenType> &tokens);
    // Expressions
    std::shared_ptr<Expression> primary();
    std::shared_ptr<Expression> unary_postfix();
    // ... Other expression production rules.
    // Statements
    std::shared_ptr<Statement> fun_declaration();
    std::shared_ptr<Statement> use_declaration();
    // ... Other statement production rules.
    public:
        // Parses a given source code and returns the code.
        void parse(
            std::shared_ptr<std::vector<std::shared_ptr<Statement>>> &code
        );
        // Creates a new parser and formats the path.
        Parser(const char *file);
        // Creates a new parser with a given formatted and initialized path.
        Parser(std::shared_ptr<const std::string> &file)
            : file(file) {}
        // ... Other helpers and debugging methods.
};
\end{minted}
\caption{Parser class}
\label{ls:parser_class}
\end{listing}

\section{Module In-memory Cache}

During the parsing of a module, this might require the use of other modules. When importing a module, the required module is first
parser. However, The module might have already been parser by a previous import, so there's no need to parse that file again. This is prevented
by having an in-memory cache to avoid re-parsing files more than once.

\begin{figure}[H]
    \centering
    \begin{tikzpicture}[node distance=2cm]
        % Nodes
        \node[state] (A) {\texttt{A}};
        \node[state, below left of=A] (B) {\texttt{B}};
        \node[state, below right of=A] (C) {\texttt{C}};
        \node[state, below left of=C] (D) {\texttt{D}};
        % Lines
        \draw[line] (A) -- (B);
        \draw[line] (A) -- (C);
        \draw[line] (B) -- (D);
        \draw[line] (C) -- (D);
    \end{tikzpicture}

    % Caption and Label
    \caption{A use case for the in-memory parser cache}
    \label{fig:module_mem_cache}
\end{figure}

This situation increases performance when a library is used by a lot of modules.
\autoref{fig:module_mem_cache} ilustrates this situation by having a module \texttt{D} imported more than once. The in-memory cache will prevent the second
parse and will assign that use node the right AST parsed previously.
