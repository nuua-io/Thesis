The job of the parser is to transform the list of tokens returned by the Lexer into an abstract syntax tree.\\

\begin{figure}[H]
    \centering
    \begin{tikzpicture}
        [node distance=1cm]

        % Nodes of the layered system
        \node[block] (input) {List of tokens};
        \node[block,right=of input] (parser) {Parser};
        \node[block,right=of lexer] (output) {Abstract syntax tree};

        % Lines
        \draw[line] (input) -- (parser);
        \draw[line] (parser) -- (output);

    \end{tikzpicture}

    % Caption and Label
    \caption{Parser overview}
    \label{fig:parser_overview}
\end{figure}

There are a lot of different algorithms and strategies for parsing grammars.
There are also a lot of parser generators out there to automatically build a parser given the grammar. However, this thesis implements
a handwritten top-down recursive descend predictive parser for the following reasons:

\begin{enumerate}
    \item To have as much control as needed over the parser. Especially for error reporting.
    \item To enforce a zero-dependency policy.
    \item Easy to build and fast enough for the job.
    \item To learn how to build a recursive descent parser and learn how it works.
\end{enumerate}

Recursive descent parsers are probably the simplest way to build a parser as mentioned in \autocite[Section~6]{crafting_interpreters}.
In fact, even if it's a simple parser it's used in very famous and large projects like \texttt{GCC} or \texttt{V8}.

As can be seen on \autocite{top_down_parsing} the top down parers can be classified into the ones using backtracking and the ones using a predictive technique.

\begin{itemize}
    \item \emph{Backtracking parsers}: Backtracking parsers try to guess the correct production rule based on the current token. This guess can, however, lead to a
        dead-end and therefore a backtrack is needed to go back and make another decision. This is, however, very inefficient for programming languages due to the number of nonterminals found in them.
    \item \emph{Predictive parsers}: Predictive parsers choose the production rule according to the current token and the next token found. The next token is called
        a lookahead token. Predictive parsers have a major issue with left-recursion and therefore, the language grammar needs to be adapted to it.
        The Nuua grammar as seen on \autoref{sec:nuua_grammar} is already adapted for left-recursion based on the solution mentioned on
        \autocite[Section~6]{crafting_interpreters}.
\end{itemize}

A recursive descent parser is a possible technique for implementing a top-down predictive parser that consists in the creation of a function for each
nonterminal found in the grammar and then calling the functions depending on the current token and the lookahead.

\section{Abstract Syntax Tree}

The abstract syntax tree (AST) is the data structure used to represent the program in memory. This tree contains the
representation of the program starting from an initial node \texttt{program} as seen in \autoref{sec:program_tld}.
\autoref{fig:ast_example} shows an example AST.

\begin{figure}[H]
    \centering
    \begin{subtable}{0.45\textwidth}
\begin{minted}{cpp}
1 - 2 * -3
\end{minted}
        \caption{Program}
    \end{subtable}
    \begin{subtable}{0.45\textwidth}
        \centering
        \texttt{TOKEN\_INTEGER TOKEN\_MINUS TOKEN\_INTEGER TOKEN\_STAR TOKEN\_MINUS TOKEN\_INTEGER}
        \caption{List of tokens}
    \end{subtable}
    \begin{subfigure}{0.45\textwidth}
        \centering
        \begin{tikzpicture}[node distance=1.75cm]
            % Nodes
            \node[state] (1) {\texttt{-}};
            \node[state, below left of=1] (2) {\texttt{1}};
            \node[state, below right of=1] (3) {\texttt{*}};
            \node[state, below left of=3] (4) {\texttt{2}};
            \node[state, below right of=3] (5) {\texttt{-}};
            \node[state, below of=5] (6) {\texttt{3}};
            % Lines
            \draw (1) -- (2);
            \draw (1) -- (3);
            \draw (3) -- (4);
            \draw (3) -- (5);
            \draw (5) -- (6);
        \end{tikzpicture}
        \caption{Abstract syntax tree}
    \end{subfigure}
    \begin{subfigure}{0.45\textwidth}
        \centering
        \begin{tikzpicture}[node distance=1.75cm]
            % Nodes
            \node[state] (0) {\texttt{P(1)}};
            \node[state, below of=0] (1) {\texttt{B(-)}};
            \node[state, below left of=1] (2) {\texttt{1}};
            \node[state, below right of=1] (3) {\texttt{B(*)}};
            \node[state, below left of=3] (4) {\texttt{2}};
            \node[state, below right of=3] (5) {\texttt{U(-)}};
            \node[state, below of=5] (6) {\texttt{3}};
            % Legend
            \node[text width=4.5cm, align=left, right = 0.5cm of 0.east] (l) {

                { \footnotesize \noindent
                    \texttt{P(n):\\\quad Program (n statements)}\\\texttt{B(operation):\\\quad Binary operation}\\\texttt{U(operation):\\\quad Unary operation}
                }
            };
            % Lines
            \draw (0) -- (1);
            \draw (1) -- (2);
            \draw (1) -- (3);
            \draw (3) -- (4);
            \draw (3) -- (5);
            \draw (5) -- (6);
        \end{tikzpicture}
        \caption{Nuua AST}
    \end{subfigure}
    \caption{Example abstract syntax tree}
    \label{fig:ast_example}
\end{figure}

All the nodes found in the abstract syntax tree of Nuua can be found at \autoref{sec:tree_nodes}. Each node is explained with the information it
stores for further use in the compiler.

\section{Value Data Types}

The parser layer is responsible for creating the type class used in the nodes and in the blocks to determine the type of the values.
Since some nodes will store information about their type and it will be added when the semantic analyzer runs. The blocks also need
the type of the variables stores so it's convenient to define it here.

The data types supported in Nuua are already explained in \autoref{sec:nuua_data_types}, therefore \autoref{ls:nuua_data_types} shows the
available data types supported with the addition of a value type that contains no type. This is used in call expression to functions that don't have
a return type. This type is only used by the compiler and not the interpreter.

\begin{listing}[H]
\begin{minted}{cpp}
// Determines the available native types in Nuua.
typedef enum : uint8_t {
    VALUE_INT, VALUE_FLOAT, VALUE_BOOL,
    VALUE_STRING, VALUE_LIST, VALUE_DICT, VALUE_FUN,
    VALUE_OBJECT, VALUE_NO_TYPE
} ValueType;
\end{minted}
\caption{Nuua data types}
\label{ls:nuua_data_types}
\end{listing}

The type class is very complex. In fact, it must store the \texttt{VALUE\_TYPE} and additional information about the type, for example,
the inner type for values like lists or dictionaries. For the function type, it must store the function parameter types and its return type. For class types,
it must store the class name used. \autoref{ls:type_class} shows the type class used in the Nuua compiler. As can be seen, there are a lot
of additional methods bound to it to work with types. Those include copy methods, reset methods, type comparisons and additionally,
methods that help to deal with the long list of casts, unary and binary operations found in \autoref{sec:expressions}.

\begin{listing}[H]
\begin{minted}{cpp}
class Type
{
    // Stores the asociation between the types
    // as string and the value type.
    static const std::unordered_map<std::string, ValueType> value_types;
    // Determines the string representation
    // of the value types.
    static const std::vector<std::string> types_string;
    public:
        // Stores the type.
        ValueType type;
        // Stores the inner type if needed.
        // Used as return type for functions.
        std::shared_ptr<Type> inner_type;
        // Class name.
        std::string class_name;
        // Store the parameters of a function type.
        std::vector<std::shared_ptr<Type>> parameters;
        // Create a value type given the type.
        Type() : type(VALUE_NO_TYPE) {}
        Type(const ValueType type)
            : type(type) {}
        // Create a type given a value type and the inner type.
        Type(
            const ValueType type,
            const std::shared_ptr<Type> &inner_type
        ) : type(type), inner_type(inner_type) {}
        // Create a type given a string representation of it.
        Type(const std::string &name);
        // Create a type given an expression and a
        // list of code blocks to know the variable values.
        Type(
            const std::shared_ptr<Expression> &rule,
            const std::vector<std::shared_ptr<Block>> *blocks
        );
        // Create a function type given the function itself.
        Type(const std::shared_ptr<Function> &fun);
        // Additional methods used as helpers for casts,
        // unary and binary operations and for other operations
        // like copy or
        // ...
};
\end{minted}
\caption{Type class}
\label{ls:type_class}
\end{listing}

\section{Block Scope and symbol table}

The parser layer will also create the block representation explained in \autoref{sec:nuua_scopes}. The way the scope block is built must be
explained here since it will be attached as part of some of the AST nodes. This acts as a symbol table for a specific scope.

A symbol table is used to store the name of the variables in the block and further verify if they are declared or get specific data of them.
For example, the symbol table can verify if a variable is declared, the type of the variable, the register where it is placed, etc.
Additionally, the block does also store the defined classes. This is done due to the fact that the block class is also used as the module scope internally.

\subsection{Block Variable Type}

The block variable type is the representation of a variable stored in the block. A variable must include the following fields:

\begin{itemize}
    \item The variable type.
    \item The AST node where it appears (The declaration).
    \item A register where this variable is stored.
    \item If the variable is exported or not.
    \item The node where it was last used.
\end{itemize}

The variable name, however, is not part of this class, since to reference this variable class, a hashmap is used and the hashmap key is a variable name.

\autoref{ls:block_variable_class} shows the class representation of a block variable type.

\begin{listing}[H]
\begin{minted}{cpp}
class BlockVariableType
{
    public:
        // Represents the variable type.
        std::shared_ptr<Type> type;
        // Stores the AST node where this variable is.
        std::shared_ptr<Node> node;
        // Represents the register where it's stored.
        register_t reg = 0;
        // Determines in the variable is exported.
        // (only applies to TLDs).
        bool exported = false;
        // Represents the last use of the variable (Variable life).
        std::shared_ptr<Node> last_use;
        // Constructors.
        BlockVariableType() {};
        BlockVariableType(
            const std::shared_ptr<Type> &type,
            const std::shared_ptr<Node> &node,
            const bool exported = false
        ) : type(type), node(node), exported(exported) {}
};
\end{minted}
\caption{BlockVariableType class}
\label{ls:block_variable_class}
\end{listing}

\subsection{Block Class Type}

The block class type is similar to the block variable type but stores much fewer fields. Also, this information is only used in the module scope.
The block class type stores the following information:

\begin{itemize}
    \item The block scope of the class.
    \item If the class is exported or not.
    \item The node where it was last used.
\end{itemize}

\autoref{ls:block_class_class} shows the class representation of the block class type. It can be seen that a forward declaration to
the block class must be used till its defined.

\begin{listing}[H]
\begin{minted}{cpp}
class Block; // Forward declaration.
class BlockClassType
{
    public:
        // Stores the block scope of the class.
        std::shared_ptr<Block> block;
        // Determines if the class is exported.
        bool exported = false;
        // Stores the AST node where this variable is.
        std::shared_ptr<Node> node;
        // Constructors.
        BlockClassType() {}
        BlockClassType(
            const std::shared_ptr<Block> &block,
            const std::shared_ptr<Node> &node,
            const bool exported = false
        ) : block(block), node(node), exported(exported) {}
};
\end{minted}
\caption{BlockClassType class}
\label{ls:block_class_class}
\end{listing}

\subsection{Block Class}

With the form of the variables and the classes that are stored in the block the final definition for a block can be given, therefore, the block must
then be able to store variables, classes and additionally provide with methods to help perform different operations on them.

\autoref{ls:block_class} shows the block class used in the Nuua compiler. As can be seen, two hashmaps are used to store the relation between
variable names and its class shown in \autoref{ls:block_variable_class} and the same with the class name with its class shown in \autoref{ls:block_class_class}.

\begin{listing}[H]
\begin{minted}{cpp}
class Block
{
    public:
        // Stores the variable name and the type of it.
        std::unordered_map<std::string, BlockVariableType> variables;
        // Stores the custom types of the block.
        std::unordered_map<std::string, BlockClassType> classes;
        // Gets a variable from the current block or returns nullptr.
        BlockVariableType *get_variable(const std::string &name);
        // Gets a class from the current block or returns nullptr.
        BlockClassType *get_class(const std::string &name);
        // Sets a variable.
        void set_variable(const std::string &name, const BlockVariableType &var);
        // Sets a class.
        void set_class(const std::string &name, const BlockClassType &c);
        // Determines if a variable is exported.
        bool is_exported(const std::string &name);
        // Determines if a class is exported.
        bool is_exported_class(const std::string &name);
        // Determines if the block have a variable.
        bool has(const std::string &name);
        // Determines if the block have a class.
        bool has_class(const std::string &name);
        // Debug the block by printing it to the screen.
        void debug() const;
        // Helper to get a single variable out of a list of blocks.
        // It iterates through it starting from the end till the front.
        static BlockVariableType *get_single_variable(const std::string &name, const std::vector<std::shared_ptr<Block>> *blocks);
};
\end{minted}
\caption{Block class}
\label{ls:block_class}
\end{listing}

\section{Parser Class}
\label{sec:parser_class}

The parser class is able to parse the program tree by using some recursive logic. When a use declaration is found when parsing,
the contents of the imported module are parsed using a new instance of the parser class. That way, the first parser class contains
the AST of all the program. It's also good to note that the parser class formats the given paths to absolute paths and checks for the
file location to determine if the imported module belongs to a relative file or the standard library. The parser class uses a similar technique
as the Lexer when dealing with the current token. The parser has a pointer to the current token being parser and the lookahead token can be
calculated by using pointer arithmetic. \autoref{ls:parser_class} shows the implementation of the parser.

\begin{listing}[H]
\begin{minted}{cpp}
class Parser
{
    // Stores the current parsing file.
    std::shared_ptr<const std::string> file;
    // Stores a pointer to the current token beeing parsed.
    Token *current = nullptr;
    // Consumes a token and returns it for futher use.
    Token *consume(const TokenType type, const std::string &message);
    // Returns true if the token type matches the current token.
    bool match(const TokenType token);
    // Returns true if any of the given token types matches the current token.
    bool match_any(const std::vector<TokenType> &tokens);
    // Expressions
    std::shared_ptr<Expression> primary();
    std::shared_ptr<Expression> unary_postfix();
    // ... Other expression production rules.
    // Statements
    std::shared_ptr<Statement> fun_declaration();
    std::shared_ptr<Statement> use_declaration();
    // ... Other statement production rules.
    public:
        // Parses a given source code and returns the code.
        void parse(
            std::shared_ptr<std::vector<std::shared_ptr<Statement>>> &code
        );
        // Creates a new parser and formats the path.
        Parser(const char *file);
        // Creates a new parser with a given formatted and initialized path.
        Parser(std::shared_ptr<const std::string> &file)
            : file(file) {}
        // ... Other helpers and debugging methods.
};
\end{minted}
\caption{Parser class}
\label{ls:parser_class}
\end{listing}

\section{In-memory Module Cache}

During the parsing of a module, this might require the use of other modules. When importing a module, the required module is first
parsed. However, The module might have already been parsed by a previous import, so there's no need to parse that file again. This is prevented
by having an in-memory cache to avoid re-parsing files more than once.

\begin{figure}[H]
    \centering
    \begin{tikzpicture}[node distance=2cm]
        % Nodes
        \node[state] (A) {\texttt{A}};
        \node[state, below left of=A] (B) {\texttt{B}};
        \node[state, below right of=A] (C) {\texttt{C}};
        \node[state, below left of=C] (D) {\texttt{D}};
        % Lines
        \draw[line] (A) -- (B);
        \draw[line] (A) -- (C);
        \draw[line] (B) -- (D);
        \draw[line] (C) -- (D);
    \end{tikzpicture}

    % Caption and Label
    \caption{A use case for the in-memory parser cache}
    \label{fig:module_mem_cache}
\end{figure}

This situation increases performance when a library is used by a lot of modules.
\autoref{fig:module_mem_cache} illustrates this situation by having a module \texttt{D} imported more than once. The in-memory cache will prevent the second
parse and will assign that use node the right AST parsed previously.
