

A node in the AST is represented by the class found \autoref{ls:node_class}

\begin{listing}[H]
\begin{minted}{cpp}
class Node
{
    public:
        // Stores the real rule of the node.
        const Rule rule;
        // Stores the file where it's found.
        std::shared_ptr<const std::string> file;
        // Stores the line where it's found.
        line_t line;
        // Stores the column where it's found.
        column_t column;
        // Constructor.
        Node(
            const Rule r,
            const std::shared_ptr<const std::string> &f,
            const line_t l, const column_t c
        ) : rule(r), file(f), line(l), column(c) {};
};
\end{minted}
\caption{Node class}
\label{ls:node_class}
\end{listing}

However, nodes are categorized into two sections. Nodes can be either statements or expressions. Therefore, two classes
are used as nodes instead as shown in \autoref{ls:expr_stmt_classes}.

\begin{listing}[H]
\begin{minted}{cpp}
class Expression : public Node
{
    public:
        explicit Expression(const Node &node)
            : Node(node) {};
};
class Statement : public Node
{
    public:
        explicit Statement(const Node &node)
            : Node(node) {};
};
\end{minted}
\caption{Expression and Statement classes}
\label{ls:expr_stmt_classes}
\end{listing}

These two classes extend the node class so they inherit all the attributes from it.

To work with all the nodes in a more effective way, the definition in \autoref{ls:node_def} is added.

\begin{listing}[H]
\begin{minted}{cpp}
#define NODE_PROPS std::shared_ptr<const std::string> &file, const line_t line, const column_t column
\end{minted}
\caption{Node properties definition}
\label{ls:node_def}
\end{listing}

\section{Integer Node}

The integer node only needs the value that corresponds to it. In this case a C++ \texttt{int64\_t} type is used as shown in
\autoref{ls:integer_node}.

\begin{listing}[H]
\begin{minted}{cpp}
class Integer : public Expression
{
    public:
        int64_t value;
        Integer(NODE_PROPS, const int64_t v)
            : Expression({ RULE_INTEGER, file, line, column }),
              value(v) {};
};
\end{minted}
\caption{Integer Node}
\label{ls:integer_node}
\end{listing}

\section{Float Node}

The float node only needs the value that corresponds to it. In this case a C++ \texttt{double} type is used as shown in
\autoref{ls:float_node}.

\begin{listing}[H]
\begin{minted}{cpp}
class Float : public Expression
{
    public:
        double value;
        Float(NODE_PROPS, const double v)
            : Expression({ RULE_FLOAT, file, line, column }),
              value(v) {};
};
\end{minted}
\caption{Float Node}
\label{ls:float_node}
\end{listing}

\section{String Node}

The string node only needs the value that corresponds to it. In this case a C++ \texttt{std::string} type is used as shown in
\autoref{ls:string_node}.

\begin{listing}[H]
\begin{minted}{cpp}
class String : public Expression
{
    public:
        std::string value;
        String(NODE_PROPS, const std::string &v)
            : Expression({ RULE_STRING, file, line, column }),
              value(v) {};
};
\end{minted}
\caption{String Node}
\label{ls:string_node}
\end{listing}

\section{Boolean Node}

The boolean node only needs the value that corresponds to it. In this case a C++ \texttt{bool} type is used as shown in
\autoref{ls:bool_node}.

\begin{listing}[H]
\begin{minted}{cpp}
class Boolean : public Expression
{
    public:
        bool value;
        Boolean(NODE_PROPS, const bool v)
            : Expression({ RULE_BOOLEAN, file, line, column }),
              value(v) {};
};
\end{minted}
\caption{Boolean Node}
\label{ls:bool_node}
\end{listing}

\section{List Node}

The list node needs the value that corresponds to it and a type parameter that's filled in by the semantic analyzer.
In this case a C++ \texttt{std::vector<std::shared\_ptr<Expression>>} type is used as shown in \autoref{ls:list_node}.
The vector contains other expressions found inside the list.

\begin{listing}[H]
\begin{minted}{cpp}
class List : public Expression
{
    public:
        std::vector<std::shared_ptr<Expression>> value;
        // Stores the list type since it's complex to analyze later.
        std::shared_ptr<Type> type;
        List(
            NODE_PROPS,
            const std::vector<std::shared_ptr<Expression>> &v
        ) : Expression({ RULE_LIST, file, line, column }),
            value(v) {};
};
\end{minted}
\caption{List Node}
\label{ls:list_node}
\end{listing}

\section{Dictionary Node}

The dictionary node needs the key-value pairs of strings and expressions and the key order. In this case a C++ \texttt{std::unordered\_map<std::string, std::shared\_ptr<Expression>>} type is used as shown in \autoref{ls:dict_node}. The type is also added and used by the semantic analyzer.

\begin{listing}[H]
\begin{minted}{cpp}
class Dictionary : public Expression
{
    public:
        std::unordered_map<std::string, std::shared_ptr<Expression>> value;
        std::vector<std::string> key_order;
        // Stores the dict type since it's complex to analyze later.
        std::shared_ptr<Type> type;
        Dictionary(
            NODE_PROPS,
            const std::unordered_map<std::string, std::shared_ptr<Expression>> &v,
            const std::vector<std::string> &ko
        ) : Expression({ RULE_DICTIONARY, file, line, column }),
            value(std::move(v)),
            key_order(std::move(ko)) {};
};
\end{minted}
\caption{Dictionary Node}
\label{ls:dict_node}
\end{listing}

\section{Group Node}

The group node consists of a single expression as seen in \autoref{ls:group_node}.

\begin{listing}[H]
\begin{minted}{cpp}
class Group : public Expression
{
    public:
        std::shared_ptr<Expression> expression;
        Group(NODE_PROPS, const std::shared_ptr<Expression> &v)
            : Expression({ RULE_GROUP, file, line, column }),
              expression(std::move(v)) {};
};
\end{minted}
\caption{Group Node}
\label{ls:group_node}
\end{listing}

\section{Unary Node}

The unary node consists of the token operation to do and the right expression. It also stores the type of unary operation as seen in
\autoref{sec:unary_expression}. \autoref{ls:unary_node} shows the representation of the unary node.

\begin{listing}[H]
\begin{minted}{cpp}
class Unary : public Expression
{
    public:
        Token op;
        std::shared_ptr<Expression> right;
        // Determines what type of unary operation
        // will be performed, no need to store a Type.
        UnaryType type = (UnaryType) NULL;
        Unary(NODE_PROPS, const Token &o, const std::shared_ptr<Expression> &r)
            : Expression({ RULE_UNARY, file, line, column }),
              op(o),
              right(std::move(r)) {};
};
\end{minted}
\caption{Unary Node}
\label{ls:unary_node}
\end{listing}

\section{Binary Node}

The binary node consists of the token operation to do and the left and right expression. It also stores the type of binary operation as seen in
\autoref{sec:binary_expression}. \autoref{ls:binary_node} shows the representation of the binary node.

\begin{listing}[H]
\begin{minted}{cpp}
class Binary : public Expression
{
    public:
        std::shared_ptr<Expression> left;
        Token op;
        std::shared_ptr<Expression> right;
        // Determines what type of binary operation will be performed.
        BinaryType type = (BinaryType) NULL;
        Binary(
            NODE_PROPS,
            const std::shared_ptr<Expression> &l,
            const Token &o,
            const std::shared_ptr<Expression> &r
        ) : Expression({ RULE_BINARY, file, line, column }),
            left(std::move(l)),
            op(o),
            right(std::move(r)) {};
};
\end{minted}
\caption{Binary Node}
\label{ls:binary_node}
\end{listing}

\section{Variable Node}

The variable node consists of the name of the variable. \autoref{ls:variable_node} shows the representation of the variable node.

\begin{listing}[H]
\begin{minted}{cpp}
class Variable : public Expression
{
    public:
        std::string name;
        Variable(NODE_PROPS, const std::string &n)
            : Expression({ RULE_VARIABLE, file, line, column }),
              name(n) {};
};
\end{minted}
\caption{Variable Node}
\label{ls:variable_node}
\end{listing}

\section{Assign Node}

The assign node consists of the target (the expression to be assigned) and the value of the assignment.
\autoref{ls:assign_node} shows the representation of the assign node. It also stores a boolean that determines if
the assignment is done to an access target (an inner value of a Nuua iterator). This boolean is used by the semantic analyzer.

\begin{listing}[H]
\begin{minted}{cpp}
class Assign : public Expression
{
    public:
        std::shared_ptr<Expression> target;
        std::shared_ptr<Expression> value;
        bool is_access = false;
        Assign(
            NODE_PROPS,
            const std::shared_ptr<Expression> &t,
            const std::shared_ptr<Expression> &v
        ) : Expression({ RULE_ASSIGN, file, line, column }),
            target(std::move(t)),
            value(std::move(v)) {};
};
\end{minted}
\caption{Assign Node}
\label{ls:assign_node}
\end{listing}

\section{Logical Node}

The logical node consists the logical operation (\texttt{or} or \texttt{and}) and the left and right expressions.
\autoref{ls:logical_node} shows the representation of the logical node.

\begin{listing}[H]
\begin{minted}{cpp}
class Logical : public Expression
{
    public:
        std::shared_ptr<Expression> left;
        Token op;
        std::shared_ptr<Expression> right;
        Logical(
            NODE_PROPS,
            const std::shared_ptr<Expression> &l,
            const Token &o,
            const std::shared_ptr<Expression> &r
        ) : Expression({ RULE_LOGICAL, file, line, column }),
            left(std::move(l)),
            op(o),
            right(std::move(r)) {};
};
\end{minted}
\caption{Logical Node}
\label{ls:logical_node}
\end{listing}

\section{Call Node}

The call node consists of the callee target and of the arguments provided to perform the call.
\autoref{ls:call_node} shows the representation of the call node. It also stores if the call has a return
depending on the callee. This helps the semantic analyzer determine where and when this call can be used since
this expression may have no value.

\begin{listing}[H]
\begin{minted}{cpp}
class Call : public Expression
{
    public:
        std::shared_ptr<Expression> target;
        std::vector<std::shared_ptr<Expression>> arguments;
        // Determines if the call target returns a value or not.
        bool has_return = false;
        Call(
            NODE_PROPS,
            const std::shared_ptr<Expression> &t,
            const std::vector<std::shared_ptr<Expression>> &a
        ) : Expression({ RULE_CALL, file, line, column }),
            target(std::move(t)),
            arguments(a) {};
};
\end{minted}
\caption{Call Node}
\label{ls:call_node}
\end{listing}

\section{Access Node}

The access node consists of the target to access and the index for it.
\autoref{ls:access_node} shows the representation of the access node.
The type represents the type of access (the type of iterator to access).

\begin{listing}[H]
\begin{minted}{cpp}
class Access : public Expression
{
    public:
        std::shared_ptr<Expression> target;
        std::shared_ptr<Expression> index;
        AccessType type = (AccessType) NULL;
        Access(
            NODE_PROPS,
            const std::shared_ptr<Expression> &t,
            const std::shared_ptr<Expression> &i
        ) : Expression({ RULE_ACCESS, file, line, column }),
            target(std::move(t)),
            index(std::move(i)) {};
};
\end{minted}
\caption{Access Node}
\label{ls:access_node}
\end{listing}

\section{Cast Node}

The cast node consists of the expression to cast and the type to cast it to.
\autoref{ls:cast_node} shows the representation of the access node.
The type of cast as shown in \autoref{sec:cast_expression} is also stored and further used by the analyzer.

\begin{listing}[H]
\begin{minted}{cpp}
class Cast : public Expression
{
    public:
        std::shared_ptr<Expression> expression;
        std::shared_ptr<Type> type;
        CastType cast_type = (CastType) NULL;
        Cast(
            NODE_PROPS,
            const std::shared_ptr<Expression> &e,
            std::shared_ptr<Type> &t
        ) : Expression({ RULE_CAST, file, line, column }),
            expression(std::move(e)),
            type(std::move(t)) {}
};
\end{minted}
\caption{Cast Node}
\label{ls:cast_node}
\end{listing}

\section{Slice Node}

The slice node consists of the expressions of the target and the expressions of the start, end and step indexes.
\autoref{ls:slice_node} shows the representation of the access node.
The additional \texttt{is\_list} represents if the slice is done to a list or a string and it's used by the semantic analyzer.

\begin{listing}[H]
\begin{minted}{cpp}
class Slice : public Expression
{
    public:
        std::shared_ptr<Expression> target;
        std::shared_ptr<Expression> start;
        std::shared_ptr<Expression> end;
        std::shared_ptr<Expression> step;
        // Determines if it's a list or a string, used by Analyzer.
        bool is_list = false;
        Slice(
            NODE_PROPS,
            const std::shared_ptr<Expression> &t,
            const std::shared_ptr<Expression> &s,
            const std::shared_ptr<Expression> &e,
            const std::shared_ptr<Expression> &st
        ) : Expression({ RULE_SLICE, file, line, column }),
            target(std::move(t)),
            start(std::move(s)),
            end(std::move(e)),
            step(std::move(st)) {}
};
\end{minted}
\caption{Slice Node}
\label{ls:slice_node}
\end{listing}

\section{Range Node}

The range node consists of the expressions of the start and end indexes.
\autoref{ls:range_node} shows the representation of the access node.
The inclusive fields indicate if the range is inclusive or exclusive.

\begin{listing}[H]
\begin{minted}{cpp}
class Range : public Expression
{
    public:
        std::shared_ptr<Expression> start;
        std::shared_ptr<Expression> end;
        bool inclusive;
        Range(
            NODE_PROPS,
            const std::shared_ptr<Expression> &s,
            const std::shared_ptr<Expression> &e,
            const bool i
        ) : Expression({ RULE_RANGE, file, line, column }),
            start(std::move(s)),
            end(std::move(e)),
            inclusive(i) {}
};
\end{minted}
\caption{Range Node}
\label{ls:range_node}
\end{listing}

\section{Delete Node}

The delete node consists of the target expression to delete.
\autoref{ls:range_node} shows the representation of the access node.

\begin{listing}[H]
\begin{minted}{cpp}
class Delete : public Expression
{
    public:
        std::shared_ptr<Expression> target;
        Delete(
            NODE_PROPS,
            const std::shared_ptr<Expression> &t
        ) : Expression({ RULE_DELETE, file, line, column }),
            target(std::move(t)) {}
};
\end{minted}
\caption{Delete Node}
\label{ls:delete_node}
\end{listing}

\section{Function Value Node}

The function value node represents a value to build a function. The node must know the function name,
the parameters of it, the return type and the body. Additionally, the scope block is also stored for further use.
\autoref{ls:function_value_node} shows the representation of the FunctionValue node.

\begin{listing}[H]
\begin{minted}{cpp}
class FunctionValue : public Expression
{
    public:
        std::string name;
        std::vector<std::shared_ptr<Declaration>> parameters;
        std::shared_ptr<Type> return_type;
        std::vector<std::shared_ptr<Statement>> body;
        std::shared_ptr<Block> block;
        FunctionValue(
            NODE_PROPS,
            const std::string &n,
            const std::vector<std::shared_ptr<Declaration>> &p,
            std::shared_ptr<Type> &rt,
            const std::vector<std::shared_ptr<Statement>> &b
        ) : Expression({ RULE_FUNCTION, file, line, column }),
            name(n), parameters(p),
            return_type(std::move(rt)),
            body(b) {}
};
\end{minted}
\caption{FunctionValue Node}
\label{ls:function_value_node}
\end{listing}

\section{Object Node}

The object node represents an object expression. The required pieces of information are the class name and the arguments to initialize the instance (as a key-value pair).
Additionally, the scope block is also stored for further use.
\autoref{ls:object_node} shows the representation of the Object node.

\begin{listing}[H]
\begin{minted}{cpp}
class Object : public Expression
{
    public:
        std::string name;
        std::unordered_map<std::string, std::shared_ptr<Expression>> arguments;
        Object(
            NODE_PROPS,
            const std::string &n,
            const std::unordered_map<std::string, std::shared_ptr<Expression>> &a
        ) : Expression({ RULE_OBJECT, file, line, column }),
            name(n),
            arguments(a) {}
};
\end{minted}
\caption{Object Node}
\label{ls:object_node}
\end{listing}

\section{Property Node}

The property node represents an access to an object property.
The information needed is the expression to access and the name of the property.
\autoref{ls:property_node} shows the representation of the Property node.

\begin{listing}[H]
\begin{minted}{cpp}
class Property : public Expression
{
    public:
        std::shared_ptr<Expression> object;
        std::string name;
        Property(
            NODE_PROPS,
            const std::shared_ptr<Expression> &o,
            const std::string &n
        ) : Expression({ RULE_PROPERTY, file, line, column }),
            object(o),
            name(n) {}
};
\end{minted}
\caption{Property Node}
\label{ls:property_node}
\end{listing}

\section{Print Node}

The print node only requires the expression to print.
\autoref{ls:print_node} shows the representation of the Print node.

\begin{listing}[H]
\begin{minted}{cpp}
class Print : public Statement
{
    public:
        std::shared_ptr<Expression> expression;
        Print(
            NODE_PROPS,
            const std::shared_ptr<Expression> &e
        ) : Statement({ RULE_PRINT, file, line, column }),
            expression(std::move(e)) {}
};
\end{minted}
\caption{Print Node}
\label{ls:print_node}
\end{listing}

\section{Expression Statement Node}

The expression statement node is used as a wrapper to treat an expression as a valid statement.
\autoref{ls:expression_statement_node} shows the representation of the ExpressionStatement node.

\begin{listing}[H]
\begin{minted}{cpp}
class ExpressionStatement : public Statement
{
    public:
        std::shared_ptr<Expression> expression;
        ExpressionStatement(
            NODE_PROPS,
            const std::shared_ptr<Expression> &e
        ) : Statement({ RULE_EXPRESSION_STATEMENT, file, line, column }),
            expression(std::move(e)) {}
};
\end{minted}
\caption{ExpressionStatement Node}
\label{ls:expression_statement_node}
\end{listing}

\section{Declaration Node}

The declaration node requires the name of the variable to declare and then a combination of the type and the initializer.
The type may be empty if the initializer is set, and the initializer may be empty if the type is set. Both of them may also be set
but not empty.
\autoref{ls:declaration_node} shows the representation of the Declaration node.

\begin{listing}[H]
\begin{minted}{cpp}
class Declaration : public Statement
{
    public:
        std::string name;
        std::shared_ptr<Type> type;
        std::shared_ptr<Expression> initializer;
        Declaration(
            NODE_PROPS,
            const std::string &n, std::shared_ptr<Type> &t,
            const std::shared_ptr<Expression> &i
        ) : Statement({ RULE_DECLARATION, file, line, column }),
            name(n),
            type(std::move(t)),
            initializer(std::move(i)) {};
};
\end{minted}
\caption{Declaration Node}
\label{ls:declaration_node}
\end{listing}

\section{Return Node}

The return node does not need an expression to be valid, but one may be provided.
\autoref{ls:return_node} shows the representation of the Return node.

\begin{listing}[H]
\begin{minted}{cpp}
class Return : public Statement
{
    public:
        std::shared_ptr<Expression> value;
        Return(
            NODE_PROPS,
            const std::shared_ptr<Expression> &v = std::shared_ptr<Expression>()
        ) : Statement({ RULE_RETURN, file, line, column }),
            value(std::move(v)) {}
};
\end{minted}
\caption{Return Node}
\label{ls:return_node}
\end{listing}

\section{If Node}

The if node does need the condition expression and the then branch followed by the then block.
Additionally, it may have an else branch and an else scope block.
\autoref{ls:if_node} shows the representation of the If node.

\begin{listing}[H]
\begin{minted}{cpp}
class If : public Statement
{
    public:
        std::shared_ptr<Expression> condition;
        std::vector<std::shared_ptr<Statement>> then_branch;
        std::vector<std::shared_ptr<Statement>> else_branch;
        std::shared_ptr<Block> then_block, else_block;
        If(
            NODE_PROPS,
            const std::shared_ptr<Expression> &c,
            const std::vector<std::shared_ptr<Statement>> &tb,
            const std::vector<std::shared_ptr<Statement>> &eb
        ) : Statement({ RULE_IF, file, line, column }),
            condition(std::move(c)),
            then_branch(tb),
            else_branch(eb) {};
};
\end{minted}
\caption{If Node}
\label{ls:if_node}
\end{listing}

\section{While Node}

The while node does need the condition expression and the body of it.
It also stores the scope block for further use.
\autoref{ls:while_node} shows the representation of the While node.

\begin{listing}[H]
\begin{minted}{cpp}
class While : public Statement
{
    public:
        std::shared_ptr<Expression> condition;
        std::vector<std::shared_ptr<Statement>> body;
        std::shared_ptr<Block> block;
        While(
            NODE_PROPS,
            const std::shared_ptr<Expression> &c,
            const std::vector<std::shared_ptr<Statement>> &b
        ) : Statement({ RULE_WHILE, file, line, column }),
            condition(std::move(c)),
            body(b) {};
};
\end{minted}
\caption{While Node}
\label{ls:while_node}
\end{listing}

\section{For Node}

The for node does require the iterator and the variable to store the loop value.
Additionally, it may have the index in case it's needed as part of the loop.
It also stores the scope block for further use.
\autoref{ls:for_node} shows the representation of the For node.

\begin{listing}[H]
\begin{minted}{cpp}
class For : public Statement
{
    public:
        std::string variable;
        std::string index;
        std::shared_ptr<Expression> iterator;
        std::vector<std::shared_ptr<Statement>> body;
        std::shared_ptr<Block> block;
        // Stores the type of the iterator.
        std::shared_ptr<Type> type;
        For(
            NODE_PROPS,
            const std::string &v,
            const std::string &i,
            const std::shared_ptr<Expression> &it,
            const std::vector<std::shared_ptr<Statement>> &b
        ) : Statement({ RULE_FOR, file, line, column }),
            variable(v),
            index(i),
            iterator(std::move(it)), body(b) {}
};
\end{minted}
\caption{For Node}
\label{ls:for_node}
\end{listing}

\section{Function Node}

The function node wraps the \texttt{FunctionValue} node as a statement.
\autoref{ls:function_node} shows the representation of the Function node.

\begin{listing}[H]
\begin{minted}{cpp}
class Function : public Statement
{
    public:
        std::shared_ptr<FunctionValue> value;
        Function(const std::shared_ptr<FunctionValue> &v)
            : Statement({ RULE_FUNCTION, v->file, v->line, v->column }),
                value(v) {}
};
\end{minted}
\caption{Function Node}
\label{ls:function_node}
\end{listing}

\section{Use Node}

The use node requires the module name to import and an optional target list. Targets can be empty and all exported targets
will be imported. It also requires a property to store the code of the module (the AST of that particular module) and
the module scope block of it.
\autoref{ls:use_node} shows the representation of the Use node.

\begin{listing}[H]
\begin{minted}{cpp}
class Use : public Statement
{
    public:
        std::vector<std::string> targets;
        std::shared_ptr<const std::string> module;
        std::shared_ptr<std::vector<std::shared_ptr<Statement>>> code;
        std::shared_ptr<Block> block;
        Use(
            NODE_PROPS,
            const std::vector<std::string> &t,
            const std::shared_ptr<const std::string> &m
        ) : Statement({ RULE_USE, file, line, column }),
            targets(t),
            module(std::move(m)) {};
};
\end{minted}
\caption{Use Node}
\label{ls:use_node}
\end{listing}

\section{Export Node}

The export node just requires the statement that is exporting.
\autoref{ls:export_node} shows the representation of the Export node.

\begin{listing}[H]
\begin{minted}{cpp}
class Export : public Statement
{
    public:
        std::shared_ptr<Statement> statement;
        Export(NODE_PROPS, const std::shared_ptr<Statement> &s)
            : Statement({ RULE_EXPORT, file, line, column }),
                statement(std::move(s)) {}
};
\end{minted}
\caption{Export Node}
\label{ls:export_node}
\end{listing}

\section{Class Node}

The class node needs the class name, the body of the class and it saves the scope block of the class for further use.
\autoref{ls:class_node} shows the representation of the Class node.

\begin{listing}[H]
\begin{minted}{cpp}
class Class : public Statement
{
    public:
        std::string name;
        std::vector<std::shared_ptr<Statement>> body;
        std::shared_ptr<Block> block;
        Class(
            NODE_PROPS,
            const std::string &n,
            const std::vector<std::shared_ptr<Statement>> &b
        ) : Statement({ RULE_CLASS, file, line, column }),
            name(n),
            body(b) {}
};
\end{minted}
\caption{Class Node}
\label{ls:class_node}
\end{listing}
