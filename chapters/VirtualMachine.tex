The virtual machine is responsible for the bytecode execution. This layer must be
carefully built since it may introduce significant bottlenecks to the program execution speed.
If the virtual machine is carefully built and the bytecode is significantly improved, the
resulting speed can be benefited. This thesis implements a register machine to interpret the code.
A stack machine could also be used but it's less efficient as described in \autocite{stack_vs_register}.

To design such a system, consideration needs to be taken at instruction dispatching. There are different ways
to dispatch instructions in a virtual machine as seen in \autocite{stack_vs_register}. The most well known and used are:

\begin{itemize}
    \item \emph{Switch dispatch}: The switch dispatch uses a (C) switch statement to decide what instruction to execute.
    \item \emph{Threaded dispatch}: The instruction dispatch is done by taking advantage of first-class labels. This is more efficient than
        a switch dispatch, but it's not part of ANSI C. There is still the possibility of using assembly to write part of the interpreter.
\end{itemize}

This thesis is designed with a C switch to be compliant with ANSI C. However, it's good to note that is not the most efficient way to dispatch instructions.

The virtual machine is responsible for storing the program. The virtual machine makes use of a program counter to decide the instruction to execute
(the program counter is implemented as a pointer to the opcode). The virtual machine must also take care of the call stack and the shared value
stack discussed in \autoref{sec:call_stack}.

The same way as the program counter points to the current instruction, two pointers are used to point to the current call frame and the top of the value stack.

\section{Virtual Machine Class}

The implementation of the virtual machine class is rather easy. The complicated stuff (memory, program structure, and frames) are already done by the
code generator, and therefore the only thing left is to implement the opcodes.

The implementation can be seen on \autoref{ls:vm_class}

\begin{listing}[H]
\begin{minted}{cpp}
#define MAX_FRAMES 1024
#define STACK_SIZE 1024
class VirtualMachine
{
    // Stores the program to be executed.
    std::shared_ptr<Program> program = std::make_shared<Program>();
    // Virtual machine program counter.
    opcode_t *program_counter;
    // Stores the virtual machine stack (used to push function arguments / get return values)
    Value stack[STACK_SIZE];
    // Indicates the top of the stack.
    Value *top_stack = this->stack;
    // Stores the current frame stack.
    Frame frames[MAX_FRAMES];
    // Indicates the top level frame.
    Frame *active_frame = this->frames - 1; // It performs a pre-increment when a call is done.
    // Runs the virtual machine.
    void run();
    public:
        // It runs the virtual machine given a source input.
        void interpret(const char *file, const std::vector<std::string> &argv);
        // Resets the virtual machine program memories.
        void reset();
};
\end{minted}
\caption{Virtual Machine Class}
\label{ls:vm_class}
\end{listing}

The two defines find on the top of \autoref{ls:vm_class} sets the size of both stacks. They can be changed.

\section{Opcode Operands Evaluation}

When the virtual machine decides to execute an opcode, this normally comes with operands. Special care needs to be taken when getting the value of the
operands. If you increment the program counter by one at a time it's fine if you do it by order but this normally requires setting some variables
to store the values. It's also good to note that C++ have unspecified order of argument evaluation and that means that things like
\texttt{a = 0; add(++a, ++a)} is not guaranteed to be \texttt{a = 0; add(1, 2)} and the compiler could chose to evaluate them in another order, for instance
\texttt{a = 0; add(2, 1)}.

If using increments + variables, the number of additions performed in an arbitrary opcode with 3 operands are:

\begin{itemize}
    \item 1 to go from the opcode to the 1st operand (program counter += 1).
    \item 1 to go from the 1st operand to the 2nd (program counter += 1).
    \item 1 to go from the 2nd operand to the 3rd (program counter += 1).
    \item 1 to fetch the next instruction (program counter += 1).
\end{itemize}

A solution can be provided by using the same amount of additions and solving problematic issues with increments and avoiding the temporary variables.

The solution requires the program counter to always point to the opcode. and further, perform additions to get the operands without modifying the program counter.

If using this method, the number of additions performed in an arbitrary opcode with 3 operands are:

\begin{itemize}
    \item 1 to fetch the 1st operand (program counter + 1).
    \item 1 to fetch the 2nd operand (program counter + 2).
    \item 1 to fetch the 3rd operand (program counter + 3).
    \item 1 to increment the program counter to the next opcode (program counter += 4)
\end{itemize}

This method also reduces the number of assignments done, since only a single assignment is performed when the program counter needs to fetch the next instruction.
