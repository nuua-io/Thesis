Although Nuua is currently in a very advanced phase, a lot of features are still missing, features that are present in most
mature programming languages. This chapter defines a set of features that are planned to be implemented into the language and
the compiler.

\begin{itemize}
    \item A C foreign function interface (FFI) to call C code using Nuua. This should allow the use of libc and other existing C code.
    \item A propper I/O interface to read and write to files, including stdin, stdout, stderr and removing the print statement.
    \item Function overloading.
    \item Extend and create a proper useful standard library.
    \item A website (\href{https://nuua.io}{nuua.io}) to announce, showcase and write all the documentation.
    \item A centralized package/module manager to automate the process of using external modules at a certain version.
    \item Use a wider string representation for the compiler. Change from \texttt{std::string} to something wider like
        \texttt{std::u16string} or \texttt{std::wstring}.
    \item Support more binary operators such as \texttt{\%, +=, -=, /= or *=}.
    \item Add more optimizations to the compiler.
\end{itemize}
