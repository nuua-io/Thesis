Programming languages are used everyday by thousands of engineers and yet few of them understand how they work.
A programming language is basically the syntax and grammar used as humans to tell a computer what to do. When
somebody wants a computer to do something, it needs to write a program using a programming language and then
a compiler needs to translate it into machine code for the computer to be able to execute it.

To design such a software system, it's vital to understand the theory behind it and learn about all the different
steps involved to make a computer understand and execute a custom programming language.

\section{Objectives}

The objectives of this document are to define and implement a fully featured experimental programming language to learn and
overcome the diferent challenges that are presented during the process. This document also explains and implement
all the different aspects of the programming language ecosystem. This includes for instance the standard library,
dependency management and a full featured webpage for the language. The experimental language is called \textbf{Nuua}.

To better summarize the objectives, this list should give a great overview to understand what this document achieves.

\begin{itemize}
    \item Understand the theory behind all steps involved in common programming languages to be able to reproduce them and adapt it
        at the needs of the project.
    \item Define a non-ambiguous grammar for the Nuua Programming Language. This grammar is simple, elegan and yet following most
        common programming language's specifications to avoid confusion when learning it, allowing either mature programmers or newcomers to
        pick up the grammar quickly and efficiently.
    \item Choose an efficient programming language to code the language with. The language would need to be efficient, fast and yet able to code with it
        pretty fast to avoid wasting time. Among other options, the languages that satify the previous statement are low level programming languages like
        C, C++, D, Rust, Go, Crystal or Nim.
    \item Define a scalable project structure to allow the programming language to grow efficiently without creating a
        big ball of mud \footnote{\href{https://en.wikipedia.org/wiki/Big_ball_of_mud}{Big ball of mud (Wikipedia)}}. The software system
        that is chosen will be very important towards the scalability of the project.
    \item Create the programming language with a defined set of features found in most programming languages. Those include from basic datatypes
        to object oriented programming.
    \item Create a Standard Library with a common set of operations or protocols implemented (For example the HTTP protocol) and also common
        libraries to work with diferent data types in a higher level of abstraction (For example data collections).
    \item Create a cloud centralized dependency manager to allow the programs to require certain libraries with a certain version. Possible
        references are NPM (Node Package Manager), Composer (PHP package manager), Cargo (Rust package manager) or Go mods (Go modules).
    \item Develop the project's website to showcase the language, store the documentation and host the information for the dependency manager.
\end{itemize}

\section{High level overview}

To firstly understand all the complex system that this document implements, it's important to explain a few key concepts
found in a typical compiler. As mentioned, the job of a compiler is to take an input source and translate it into
another. Often, the compiler term is used to express a translation to a much diferent environment, that means that usually, the input
is written in a high level language and further translated into a lower level of abstraction. As an example, GCC (GNU Compiler Collection) and
it's implementation of the C programming language basically compiles the input code into assembly (and optimized if needed), then to an object file and finally linked
to generate an executable. That said, the compiler job was to translate a high level programming language (C) into a low level programming language (Assembly).
However, there are multiple types of compilers, and each one is used to implement different translation tasks.

\begin{itemize}
    \item \textbf{Compiler}: Translates a high level language to a low level language in the same or diferent architecture.
        For example, GCC's C implementation
    \item \textbf{Cross-Compiler}: The translation is done to a specific (and probably diferent than the compiler's machine) architecture.
        For example, AVR-GCC compiles specifically for the AVR microcontroller architecture. Other closer examples may be found when compiling for
        32 bits or 64 bits target architecture.
    \item \textbf{Bootstrap compiler}: A compiler that is built using the language it compiles to (The initial version is written in another language)
        For example, the official Go compiler is written in Go.
    \item \textbf{Decompiler}: Translates a low level language to a high level language in the same or diferent architecture.
        For example, the Boomerang decompiler.
    \item \textbf{Transpiler (Source-to-source compiler)}: Translates a language to a similar level of abstraction (usually between high level languages).
        For example, TypeScript (compiles to JavaScript).
\end{itemize}

\begin{figure}[H]
    \centering
    \begin{tikzpicture}
        [node distance=1cm]

        % Nodes of the layered system
        \node[block] (input) {Input language};
        \node[block,right=of input] (compiler) {Compiler};
        \node[block,right=of compiler] (output) {Output language};

        % Lines
        \draw[line] (input) -- (compiler);
        \draw[line] (compiler) -- (output);

    \end{tikzpicture}

    % Caption and Label
    \caption{Compiler overview}
    \label{fig:compiler_overview}
\end{figure}

\section{Language runtime}

A compiler job is just to translate, and then, it's up to the language implementation to run it. There are diferent ways to
run or execute a programming language.

\begin{itemize}
    \item \textbf{Machine-language compiler}: If the target language of the compiler is executable code, this can be run
        by the operating system and it will produce the nessesary outputs. Machine code have a big advantage and disadvantage.
        The executed code will be very fast but the executable is not portable among diferent architectured. This introduces
        platform-dependent builds that are very fast.
    \item \textbf{Interpreter}: Interpreters are often a very diferent overview of this problem. Interpreters do not compile
        to machine language, and instead, it often compiles to bytecode (a portable and reduced instruction set similar to machine instructions
        that is used to execute the language. The advantages are that is platform-independent and it's designed specifically with the language
        needs but one big disadvantage is that it's much slower to run than real machine-code. Interpreters can also make language execution
        dynamic allowing untyped languages that plays around with diferent variable types.
    \item \textbf{Just-in-time compiler}: Just-in-time compilers (often called JIT compilers) are an intermidiate aproach between the other two
        that tries to speed up the execution of interpreters by compiling chunks of code at a specific moment while the program run to speed up portions
        of the code that is often executed (for example functions that are called frequently). Esentially, the JIT compiler needs to decide when to compile
        a specific part of the code at runtime and adds a small overhead in exchange for a machine-language performance on successive calls.
\end{itemize}

\section{Structure}

Nuua is structured as a layered system. This system is known to be robust and scalable at very low
performance cost. It's very easy to understand and yet very powerful for software systems like this one.
This systems are also used for other complex software systems, such as operating systems or complex
protocols like TCP/IP.

By using a layered system, you can make sure each layer's function is correctly executed without
messing with the whole system itself, creating a very impresive way to scale-up or upgrade existing
parts of the system without damaging the others. However, a consistent API should in fact be established
from the ground up to avoid breaking changes. If the API is maintained, the individual layers may be
upgraded independently without the need of extra work. A layered system consists of diferent independent
ordered layers that need to meet certain requirements. The requirements are mostly simple, each layer can
only use the API given by the layer that's below it and it can only give an API to the one above.

Since this structure is ideal for Nuua, the decision to build it upon it was inevitable. Therefore, the layers
Nuua have in it's system are exposed in \autoref{fig:nuua_system}. An independent module called Logger is found
on the left side of the figure. This module is in fact a simple logger used by all layers to output messages if
needed (like errors messages). However, since all modules require the use of it, it can be used but not modified.

There's also three distinct sections found in the system, the back end, the middle end and the front end. I'll talk about them in
a future section since it's not the purpose of the current one. However it's important to keep in mind they exist.

\begin{figure}[H]
    \centering
    \begin{tikzpicture}
        [node distance=1cm]

        % Nodes of the layered system
        \node[big_block] (app) {Application};
        \node[big_block,below=of app] (vm) {Virtual Machine};
        \node[big_block,below=of vm] (compiler) {Compiler};
        \node[big_block,below=of compiler] (analyzer) {Analyzer};
        \node[big_block,below=of analyzer] (parser) {Parser};
        \node[big_block,below=of parser] (lexer) {Lexer};

        % Independent module
        \node[block, left=of compiler] (logger) {Logger};

        % Separator
        \draw[thick,dashed] ($(app.north west) + (-0.5, 0)$) -- ($(lexer.south west) + (-0.5, 0)$);

        % Top arrows
        \draw[line] (-3, 1.75) node[anchor=east] {Input} -| ($(app.north) + (-1, 0)$);
        \draw[line] ($(app.north) + (1, 0)$) |- (3, 1.75) node[anchor=west] {Output};

        % Arrows going down
        \draw[line] ($(app.south) + (-1, 0)$) -- ($(vm.north) + (-1, 0)$);
        \draw[line] ($(vm.south) + (-1, 0)$) -- ($(compiler.north) + (-1, 0)$);
        \draw[line] ($(compiler.south) + (-1, 0)$) -- ($(analyzer.north) + (-1, 0)$);
        \draw[line] ($(analyzer.south) + (-1, 0)$) -- ($(parser.north) + (-1, 0)$);
        \draw[line] ($(parser.south) + (-1, 0)$) -- ($(lexer.north) + (-1, 0)$);

        % Arrows going up
        \draw[line] ($(vm.north) + (1, 0)$) -- ($(app.south) + (1, 0)$);
        \draw[line] ($(compiler.north) + (1, 0)$) -- ($(vm.south) + (1, 0)$);
        \draw[line] ($(analyzer.north) + (1, 0)$) -- ($(compiler.south) + (1, 0)$);
        \draw[line] ($(parser.north) + (1, 0)$) -- ($(analyzer.south) + (1, 0)$);
        \draw[line] ($(lexer.north) + (1, 0)$) -- ($(parser.south) + (1, 0)$);

        % Logger arrows
        \draw[line] (app.west) -- (logger);
        \draw[line] (vm.west) -- (logger);
        \draw[line] (compiler.west) -- (logger);
        \draw[line] (analyzer.west) -- (logger);
        \draw[line] (parser.west) -- (logger);
        \draw[line] (lexer.west) -- (logger);

        % The right specifiers
        \draw[tuborg, decoration={brace}] let \p1=(parser.north), \p2=(lexer.south) in
            ($(2.5, \y1)$) -- ($(2.5, \y2)$) node[tubnode] {Front end};
        \draw[tuborg, decoration={brace}] let \p1=(analyzer.north), \p2=(analyzer.south) in
            ($(2.5, \y1)$) -- ($(2.5, \y2)$) node[tubnode] {Middle end};
        \draw[tuborg, decoration={brace}] let \p1=(vm.north), \p2=(compiler.south) in
            ($(2.5, \y1)$) -- ($(2.5, \y2)$) node[tubnode] {Back end};
    \end{tikzpicture}

    % Caption and Label
    \caption{Nuua's Layered System}
    \label{fig:nuua_system}
\end{figure}
