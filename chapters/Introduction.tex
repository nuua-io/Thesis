Programming languages are used every day by millions of engineers as part of their daily routine.
A programming language is used to tell a computer what to do. When
somebody wants a computer to do something, it needs to write a program using a programming language. Then,
a compiler needs to translate it into machine code to be executed.

To design a programming language it's important to understand the theory behind a compiler and to learn about all the
steps involved to make a computer understand and execute a program.

\section{Objectives}

The main objective of this thesis is to design an experimental programming language and implement an interpreter to execute
any program written with it. The different challenges that are faced during the design and implementation process are also explained and
solved in their respective chapters. The experimental language built in this thesis is called \emph{Nuua}.

The objective can be partitioned into the following points.

\begin{itemize}
    \item Learn all the steps involved in a common compiler implementation and reproduce them according to the project needs.
    \item Design the Nuua Programming Language. The grammar must be simple, elegant and yet it needs to follow the most
        common programming language's specifications to have a low learning curvature.
    \item Choose an efficient programming language to build the compiler and the interpreter with. Among other options, the languages that
    satisfy the previous statement are low-level programming languages like C \autocite{c_programming_language}, C++ \autocite{cpp_programming_language},
    D \autocite{d_programming_language}, Rust \autocite{rust_programming_language} or Go \autocite{go_programming_language} among others.
    \item Define a robust system architecture to design the compiler and the interpreter. The system architecture needs to be scalable.
    \item Build a compiler and an interpreter for the Nuua programming language and a very simple standard library.
\end{itemize}

\section{Preliminary overview}

This section briefly introduces some of the concepts found in this thesis, introducing preliminary concepts of language grammar, compilers and interpreters.
This preliminary overview won't deal with details and only explains the basics to understand the whole system without deep knowledge.
Further chapters contain expanded information respective to some of the details mentioned here.

\subsection{Language grammar}

Context-free grammar is a notation used to specify the syntax of a programming language. Following the syntax definition explained
in \autocite[Section~2.2]{compilers} a context-free grammar consists of four components:

\begin{enumerate}
    \item A group of terminal symbols also known as tokens. In a programming language tokens may be literal symbols like '+', '*' or numbers and identifiers.
    \item A group of non-terminals that can be reduced to terminals based on the production rules.
    \item A group of production rules that consists of a non-terminal on the left side and a sequence of terminals and/or non-terminals on the right side.
    \item A non-terminal start symbol.
\end{enumerate}

This thesis will use the \emph{extended Backus-Naur form} also known as \emph{EBNF} to express the context-free grammar representation of Nuua.
EBNF is often used in different ways due to the big amount of variants that exist. To EBNF syntax that this thesis is uses is shown in the
\autoref{fig:ebnf_syntax}. More information may be found in the EBNF grammar article \autocite{EBNF_grammar}.

\begin{table}[H]
    \centering
    \begin{tabular}{ l p{10cm} }
        \textbf{Symbol} & \textbf{Definition} \\
        \texttt{:} & Used to define a production rule. \\
        \texttt{\textit{space}} & Used to concatenate patterns (space separated). \\
        \texttt{A|B} & Used to define a union of A and B. \\
        \texttt{A+} & Used to define a one or more pattern of A. \\
        \texttt{A*} & Used to define a zero or more pattern. \\
        \texttt{A?} & Used to define an optional pattern. \\
        \texttt{(A)} & Used to group a pattern. \\
        \texttt{"T"} or \texttt{'T'} & Used to define a terminal symbol. \\
        \texttt{@A} & Used to indicate anything except A. \\
        \texttt{;} & Used to terminate a given production rule. \\
    \end{tabular}
    \caption{Variation of EBNF syntax used by this thesis}
    \label{fig:ebnf_syntax}
\end{table}

As a simple example, to define a language that consists of a single integer, the following EBNF grammar could be used:\\

\texttt{integer\\\tab: ("-" | "+")? digit+\\\tab;}\\
\texttt{digit\\\tab: "0"|"1"|"2"|"3"|"4"|"5"|"6"|"7"|"8"|"9"\\\tab;}\\

This variation is often used by many parser generators since it introduces a more visible and versatile approach to
write the language grammar.

\subsection{Compilers}

The job of a compiler is to take an input program written in a programming language and translate it into another as shown in \autoref{fig:compiler_overview}.
The compiler term is often used to express a translation to a much different level of abstraction, that usually means
that the input is written in a high-level language and further translated into another low-level language.

\begin{figure}[H]
    \centering
    \begin{tikzpicture}
        [node distance=1cm]

        % Nodes of the layered system
        \node[block] (input) {Input};
        \node[block,right=of input] (compiler) {Compiler};
        \node[block,right=of compiler] (output) {Output};

        % Lines
        \draw[line] (input) -- (compiler);
        \draw[line] (compiler) -- (output);

    \end{tikzpicture}

    % Caption and Label
    \caption{Compiler overview}
    \label{fig:compiler_overview}
\end{figure}

\subsubsection{Phases of a compiler}

A compiler can also be decoupled into different parts. Each part does a very different job but they are all connected to each other.
In a typical compiler architecture, we may find all the different phases described in \autoref{fig:compiler_phases}.

Those phases are often found to be different depending on the implementation of the language. However, it's important to note what they do,
since they are often implemented in one way or another. More implementation details are explained in their respective chapters but in this
section, a small introduction to each phase is needed to understand the Nuua's system.

\begin{figure}[p]
    \centering
    \begin{tikzpicture}
        [node distance=1cm]

        \node (input) {Input};
        \node[bigger_block,below=of input] (lexical) {Lexical analysis (Lexer or scanner)};
        \node[bigger_block,below=of lexical] (syntax) {Syntax analysis (Parser)};
        \node[bigger_block,below=of syntax] (semantic) {Semantic analysis};
        \node[bigger_block,below=of semantic] (inter) {Intermediate code generator};
        \node[bigger_block,below=of inter] (opt) {Optimization};
        \node[bigger_block,below=of opt] (codegen) {Code generation};
        \node[below=of codegen] (output) {Output};

        \draw[line] (input) -- (lexical);
        \draw[line] (lexical) -- (syntax);
        \draw[line] (syntax) -- (semantic);
        \draw[line] (semantic) -- (inter);
        \draw[line] (inter) -- (opt);
        \draw[line] (opt) -- (codegen);
        \draw[line] (codegen) -- (output);

        % The right specifiers
        \draw[tuborg, decoration={brace}] let \p1=(lexical.north), \p2=(semantic.south) in
            ($(3.5, \y1)$) -- ($(3.5, \y2)$) node[tubnode] {Front end};
        \draw[tuborg, decoration={brace}] let \p1=(inter.north), \p2=(opt.south) in
            ($(3.5, \y1)$) -- ($(3.5, \y2)$) node[tubnode] {Middle end};
        \draw[tuborg, decoration={brace}] let \p1=(codegen.north), \p2=(codegen.south) in
            ($(3.5, \y1)$) -- ($(3.5, \y2)$) node[tubnode] {Back end};
    \end{tikzpicture}

    % Caption and Label
    \caption{Common compiler phases}
    \label{fig:compiler_phases}
\end{figure}

\begin{itemize}
    \item \emph{Lexical analysis}: In this phase, the input source is transformed from a character string into a token list, this is also called
        tokenization. Lexemes found in the source program are translated into individual tokens using different patterns. For example, some tokens
        might include integers, symbols (+, -, *, etc.), identifiers or keywords ('if', 'while', etc.).
    \item \emph{Syntax analysis}: In this phase, the implementation may vary among compilers, some of them work close to the lexical analysis since they
        can work together. However, its purpose is to perform operations given the token list to parse the input and create an Abstract Syntax Tree
        (AST). As seen in \autocite[Section~5.2.1]{engineering_a_compiler}, an AST is a data structure that represents the input program.
        (AST). As seen in \autocite[Section~5.2.1]{engineering_a_compiler}, an AST is a data structure that represents the input program.
        This stage determines if it's a valid program based on the language grammar and the specified rules.
        There are also scanner-less parsers that take the lexical analysis and the syntax analysis into a single step. It is harder to understand and
        debug compared to the modularity of splitting these two phases but it has some advantages like removing the token classification as mentioned
        in \autocite{scannerless_parsing}.
    \item \emph{Semantic analysis}: This phase analyzes the AST and creates a symbol table while analyzes the input source with things like type checking or
        variable declarations, if some operations can be performed (for example adding a number with a string). A symbol table is a structure used
        by further phases to see information attached to specific source code parts. For example, it can store information about a variable
        (if it's global, exported, etc.).
    \item \emph{Intermediate code generator}: An Intermediate representation (IR) can be avoided but it's often used to have a platform independent
        optimizer. Usually the code generation targets a specific architecture and would require different optimizers depending on each architecture.
        However, by having an IR it's possible to have a single optimizer. A much used IR is Three Address Code
        (TAC) that can be organized in quadruples or triples as seen by some examples in \autocite{three_address_code_examples}.
    \item \emph{Optimization}: This optimization is often performed on the IR and performs different tasks to allow a faster and smaller output. For example, it may
        remove dead code, perform loop optimizations, etc.
    \item \emph{Code generation}: This is where the real translation takes place, it translates the IR into a different language output. For example machine code.
        This phase often has to deal with instruction scheduling or register allocation while they have to output a fully working program.
\end{itemize}

\subsection{Interpreters}

There are different ways to interpret a program. Among the most popular options we can find a bytecode interpreter and an AST interpreter.

\begin{itemize}
    \item \emph{Bytecode interpreter}: The program is first compiled to bytecode instructions and further interpreted. Bytecode interpreters are often
        implemented as virtual machines since most of the times bytecode instructions are very similar to real hardware instructions.
        The usual choices are stack or register machines. There have been many discussions on the advantages and the inconveniences of both of them, more information may be seen at \autocite{stack_vs_register}. An example of a virtual machine is the Lua virtual machine \autocite{the_implementation_of_lua}.
    \item \emph{Abstract syntax tree interpreter}: This kind of interpreters just need the AST to work with, so no extra compilation to bytecode
        is needed and therefore, they are easier to implement. However, due it's nature, they are much slower to execute and debug due to the recursiveness of working with tree data structures.
\end{itemize}

\subsection{Just-in-time compilers}

Just-in-time compilers (often called JIT compilers) are an intermediate approach between a compiler that generates machine code and an interpreter.
JIT compilers compile chunks of code at specific moments while the program run to speed up portions of the code that is being interpreted
(for example functions that are called frequently). Essentially, the JIT compiler needs to decide when to compile a specific part of the code at
runtime and adds a small overhead in exchange for a machine-language performance on specific parts of the program.

More information about JIT compilers and the tools that can be used can be found at \autocite{jit_compilation}.

\section{System architecture}
\label{sec:sys_arch}

To design the system architecture of Nuua, consideration of existing system architectures needs to be taken since existing architectures
often work better than the custom-made ones and they often lead to greater project scalability. It's trivial
to make this choice before starting the project since changing a system architecture after it's initial
development phases becomes a very bad choice and may lead to a big ball of mud. Two choices in software development might be hierarchical
or layered systems. Nuua's architecture is based on a \emph{layered system} \autocite{software_architecture}.

A layered system has specific requirements regarding code communication. Specifically, a layered system consists
of different layers arranged vertically. Those layers have a specific criterion that needs to be met. As a matter of fact,
each layer can only use the layer below and gives an API for the layer above (if any) to use its functions. For example,
the \autoref{fig:layered_system} shows a simple 3-tier layered system. Layer 3 can only use Layer 2, and the output comes
from Layer 2. Layer 3 cannot use Layer 1 nor expect any outputs from it. It's the Layer 2 responsibility to use the Layer 1
and process its output before it can give its own output.

\begin{figure}[H]
    \centering
    \begin{tikzpicture}
        [node distance=1.25cm]

        % Nodes of the layered system
        \node[block] (layerc) {Layer 3};
        \node[block,below=of layerc] (layerb) {Layer 2};
        \node[block,below=of layerb] (layera) {Layer 1};

        % Arrows going down
        \draw[line] ($(layerc.south) + (-1, 0)$) -- node[midway, left] {Uses} ($(layerb.north) + (-1, 0)$);
        \draw[line] ($(layerb.south) + (-1, 0)$) -- node[midway, left] {Uses} ($(layera.north) + (-1, 0)$);

        % Arrows going up
        \draw[line] ($(layerb.north) + (1, 0)$) -- node[midway, right] {Output} ($(layerc.south) + (1, 0)$);
        \draw[line] ($(layera.north) + (1, 0)$) -- node[midway, right] {Output} ($(layerb.south) + (1, 0)$);

    \end{tikzpicture}

    % Caption and Label
    \caption{Layered system}
    \label{fig:layered_system}
\end{figure}

This system is known to be robust, easy to test and with a high ease of development as mentioned in \autocite{software_architecture_patterns}.
It's very easy to understand and a widely used system. This system is also used for other complex software systems,
such as operating systems or complex protocols like TCP/IP.

By using a layered system each layer gets completely isolated and works independently by just using the layer below,
creating a way to scale-up or upgrade existing parts of the system without damaging the others. This introduces a
very powerful \emph{separation of concerns} among all the system layers since each layer has a specific role and only deals
with the logic that pertains to it.
However, a consistent API should, in fact, be established from the ground up to avoid backward incompatible changes.
If the API is maintained, the individual layers may be upgraded independently without the need for extra work.

\autoref{fig:nuua_system} shows the Nuua architecture. An independent module called Logger is found
on the left side of the figure. This module is a logger used by all layers to output messages if needed (for example error reporting).

\begin{figure}[p]
    \centering
    \begin{tikzpicture}
        [node distance=1cm]

        % Nodes of the layered system
        \node[big_block] (app) {Application};
        \node[big_block,below=of app] (vm) {Virtual Machine};
        \node[big_block,below=of vm] (compiler) {Code generator};
        \node[big_block,below=of compiler] (analyzer) {Semantic analyzer};
        \node[big_block,below=of analyzer] (parser) {Parser};
        \node[big_block,below=of parser] (lexer) {Lexer};

        % Independent module
        \node[block, left=of compiler] (logger) {Logger};

        % Separator
        \draw[thick,dashed] ($(app.north west) + (-0.5, 0)$) -- ($(lexer.south west) + (-0.5, 0)$);

        % Top arrows
        \draw[line] (-3, 1.75) node[anchor=east] {Input} -| ($(app.north) + (-1, 0)$);
        \draw[line] ($(app.north) + (1, 0)$) |- (3, 1.75) node[anchor=west] {Output};

        % Arrows going down
        \draw[line] ($(app.south) + (-1, 0)$) -- ($(vm.north) + (-1, 0)$);
        \draw[line] ($(vm.south) + (-1, 0)$) -- ($(compiler.north) + (-1, 0)$);
        \draw[line] ($(compiler.south) + (-1, 0)$) -- ($(analyzer.north) + (-1, 0)$);
        \draw[line] ($(analyzer.south) + (-1, 0)$) -- ($(parser.north) + (-1, 0)$);
        \draw[line] ($(parser.south) + (-1, 0)$) -- ($(lexer.north) + (-1, 0)$);

        % Arrows going up
        \draw[line] ($(vm.north) + (1, 0)$) -- ($(app.south) + (1, 0)$);
        \draw[line] ($(compiler.north) + (1, 0)$) -- ($(vm.south) + (1, 0)$);
        \draw[line] ($(analyzer.north) + (1, 0)$) -- ($(compiler.south) + (1, 0)$);
        \draw[line] ($(parser.north) + (1, 0)$) -- ($(analyzer.south) + (1, 0)$);
        \draw[line] ($(lexer.north) + (1, 0)$) -- ($(parser.south) + (1, 0)$);

        % Logger arrows
        \draw[line] (app.west) -- (logger);
        \draw[line] (vm.west) -- (logger);
        \draw[line] (compiler.west) -- (logger);
        \draw[line] (analyzer.west) -- (logger);
        \draw[line] (parser.west) -- (logger);
        \draw[line] (lexer.west) -- (logger);

        % The right specifiers
        \draw[tuborg, decoration={brace}] let \p1=(app.north), \p2=(vm.south) in
            ($(2.5, \y1)$) -- ($(2.5, \y2)$) node[tubnode] {Interpreter};
        \draw[tuborg, decoration={brace}] let \p1=(compiler.north), \p2=(lexer.south) in
            ($(2.5, \y1)$) -- ($(2.5, \y2)$) node[tubnode] {Compiler};
    \end{tikzpicture}

    % Caption and Label
    \caption{Nuua's architecture diagram (Layered System)}
    \label{fig:nuua_system}
\end{figure}

\begin{itemize}
    \item \emph{Logger}: Used by all layers to debug or log errors. In case of a fatal error, the logger outputs
        an error stack in a fancy way and terminates the application.
    \item \emph{Application}: This layer is used to decode the command line arguments and fire up the compiler toolchain. In short,
        the purpose is to analyze the command line arguments and fire the application accordingly.
    \item \emph{Virtual Machine}: The virtual machine is the interpreter that runs Nuua. It's a register-based virtual machine that
        acts as the Nuua runtime environment.
    \item \emph{Code generator}: Is responsible for the translation of the AST to the virtual machine bytecode. This acts as the code generation
        part of a compiler architecture.
    \item \emph{Semantic analyzer}: Does all the semantic analysis of the compiler and optimizes the AST for faster and smaller output.
    \item \emph{Parser}: Acts as the syntax analyzer, it uses a list of tokens and generates a fully valid AST.
    \item \emph{Lexer}: Scans the source code and translates the characters to tokens, creating a list of tokens representing
        the original source code.
\end{itemize}
