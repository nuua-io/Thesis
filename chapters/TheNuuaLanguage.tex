This chapter defines the Nuua programming language. The language is set to be of a general purpose usage, with an inperative
paradigm and a statically typed system. The official Nuua compiler is written in C++ with a zero-dependency policy and it's
implemented as an interpreter using a register-based virtual machine.

\section{Nuua Grammar}

Nuua's grammar is inspired by other existing programming languages by taking advantage of some of the best features some of them offer.
Inspiration comes specially from Python, Rust and Go.

The precedence relationship between expressions is heavily inspired by C, D, Rust and Dart. The official documentation for those languages
exposes similar tables for operator precedences and Nuua has taken akin levels of precedence as those.

Nuua does not make use of the \texttt{';'} to separate statements, instead, it uses the same separator as Go. Statements can be separated by
a \texttt{'\textbackslash n'} but it does not make use of the \texttt{\textbackslash t} to indicate statements inside blocks, and uses the typical
block separator \texttt{\{ ... \}}.

\subsection{Lexical Grammar}

The lexical grammar is used by the lowest layer of the Nuua system to scan the source language and identify different terminal symbols.
The main difference with the syntax grammar, as exposed in \autocite[Appendix~I]{crafting_interpreters} is that the syntax grammar is a
context free grammar and the lexical grammar is a regular grammar.

Nuua's lexical rules are as follors:

\texttt{
    DIGIT\\
        \tab: '0' ... '9'\\
        \tab;
}

Digits are any character from '0' to '9', given that the ASCII table \footnote{\href{https://en.wikipedia.org/wiki/ASCII}{ASCII table (Wikipedia)}} uses
a sequential enumeration for them, it's very easy to determine what characters are between '0' and '9'.

\texttt{
    ALPHA\\
        \tab: 'a' ... 'z'\\
        \tab| 'A' ... 'Z'\\
        \tab| '\_'\\
        \tab;
}

Alpha are characters that are part of the english alphabet in lower or upper case. It also includes '\_' as a special character.

\texttt{
    ALPHANUM\\
        \tab: DIGIT\\
        \tab| ALPHA\\
        \tab;
}

Alphanum are characters that are either part of the alphabet or are digits.

\texttt{
    INTEGER\\
        \tab: DIGIT+\\
        \tab;
}

Integers are a single digit or more found sequentially without spaces. The integer sign is not represented here.

\texttt{
    FLOAT\\
        \tab: DIGIT+ '.' DIGIT+\\
        \tab;
}

Floats are like integers but require a dot followed by a digit or more, creating a decimal number.

\texttt{
    BOOL\\
        \tab: 'true'\\
        \tab| 'false'\\
        \tab;
}

Bools are either 'true' or 'false', that are reserved words.

\texttt{
    STRING\\
        \tab: '"' \# '"' \# '"'\\
        \tab;
}

Strings represent a character string with the possibility to escape '"' by using a '\textbackslash' as a prefix, more on that in the upcomming sections.

\texttt{
    IDENTIFIER\\
        \tab: ALPHA ALPHANUM*\\
        \tab;
}

Identifiers are part an alpha character followed by an optional one or more alphanumeric character.

\subsection{Syntax Grammar}

\subsubsection{Program and Top Level Declarations}

\texttt{
    program\\
        \tab: top\_level\_declaration*\\
        \tab;
}

A Nuua program is a list of top level declarations.

\texttt{
    top\_level\_declaration\\
        \tab: use\_declaration '\textbackslash n'\\
        \tab| export\_declaration '\textbackslash n'\\
        \tab| class\_declaration '\textbackslash n'\\
        \tab| fun\_declaration '\textbackslash n'\\
        \tab;
}

A top level delcaration can only be one of the specified rules. Top level declarations are a
special type of declaration that can only be declared on the module and not inside other blocks.

\texttt{
    export\_declaration\\
        \tab: "export" top\_level\_declaration\\
        \tab;
}

An export declaration marks the following top level declaration as exported, making it available for other modules to import it using the
use declaration.

\texttt{
    use\_declaration\\
        \tab: "use" STRING\\
        \tab| "use" IDENTIFIER ("," IDENTIFIER)* "from" STRING\\
        \tab;
}

A use declaration is used to import other top level declarations from other modules. By using the first rule, Nuua imports all the
exported targets of the module pointed by STRING. Otherwise, Nuua imports the specified targets from the modules.

\texttt{
    class\_declaration\\
        \tab: "use" STRING\\
        \tab| "use" IDENTIFIER ("," IDENTIFIER)* "from" STRING\\
        \tab;
}

\subsubsection{Expressions}

\texttt{
    LIST\\
        \tab: "[" expression (',' expression)* "]"\\
        \tab;
}

Lists can't be empty, so a at least one expression must be provided.

\texttt{
    DICTIONARY\\
        \tab: "\{" IDENTIFIER ':' expression (',' IDENTIFIER ':' expression)* "\}"\\
        \tab;
}

Dictionaries, as lists, can't be empty, so a at least one expression must be provided.

\subsection{Operator precedence}
